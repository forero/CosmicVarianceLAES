\documentclass{emulateapj}
%\documentclass{aastex}
\submitted{{\it Submitted for publication in ApJL}}
\usepackage {graphicx}

\usepackage{amsmath}
\usepackage{amssymb}
\usepackage{graphics}
\usepackage{epsfig}
\usepackage{float}
\bibliographystyle{apj}
\def\be{\begin{equation}}
\def\ee{\end{equation}}
\def\ba{\begin{eqnarray}}
\def\ea{\end{eqnarray}}


\newcommand{\documentname}{paper~}
\newcommand{\match}{{\tt match}~}
\newcommand{\kms}{\,km~s$^{-1}$}
\def\squig{\sim\!\!}
\newcommand{\LCDM}{$\Lambda$CDM~}
\newcommand{\beq}{\begin{eqnarray}}  
\newcommand{\eeq}{\end{eqnarray}}   
\newcommand{\zz}{$z\sim 3$} 
\newcommand{\avg}[1]{\langle{#1}\rangle}  
\newcommand{\ly}{{\ifmmode{{\rm Ly}\alpha}\else{Ly$\alpha$~}\fi}}
\newcommand{\hMpc}{{\ifmmode{h^{-1}{\rm Mpc}}\else{$h^{-1}$Mpc }\fi}}  
\newcommand{\hGpc}{{\ifmmode{h^{-1}{\rm Gpc}}\else{$h^{-1}$Gpc }\fi}}  
\newcommand{\hmpc}{{\ifmmode{h^{-1}{\rm Mpc}}\else{$h^{-1}$Mpc }\fi}}  
\newcommand{\hkpc}{{\ifmmode{h^{-1}{\rm kpc}}\else{$h^{-1}$kpc }\fi}}  
\newcommand{\hMsun}{{\ifmmode{h^{-1}{\rm
        {M_{\odot}}}}\else{$h^{-1}{\rm{M_{\odot}}}$}\fi}}   
\newcommand{\hmsun}{{\ifmmode{h^{-1}{\rm
        {M_{\odot}}}}\else{$h^{-1}{\rm{M_{\odot}}}$}\fi}}   
\newcommand{\Msun}{{\ifmmode{{\rm {M_{\odot}}}}\else{${\rm{M_{\odot}}}$}\fi}}  
\newcommand{\msun}{{\ifmmode{{\rm {M_{\odot}}}}\else{${\rm{M_{\odot}}}$}\fi}}  
\newcommand{\clara}{{\texttt{CLARA}}~}
\newcommand{\rand}{{\ifmmode{{\mathcal{R}}}\else{${\mathcal{R}}$ }\fi}}  
\newcommand{\Lsun}{\mbox{\,$L_{\odot}$}}
\newcommand{\like}{\mathscr{L}}
\newcommand{\bftheta}{\mathbf{\Theta}}
\newcommand{\degree}{\ensuremath{^\circ}}
\def\spose#1{\hbox to 0pt{#1\hss}}
\def\simlt{\mathrel{\spose{\lower 3pt\hbox{$\mathchar"218$}}
     \raise 2.0pt\hbox{$\mathchar"13C$}}}
\def\simgt{\mathrel{\spose{\lower 3pt\hbox{$\mathchar"218$}}
     \raise 2.0pt\hbox{$\mathchar"13E$}}}
\font\smcap=cmcsc10

\begin{document}


\title{The impact of cosmic variance on constraining the mass and
  occupation fraction of dark matter halos hosting Lyman-$\alpha$
  emitters at $z \sim 3$}      
\author{
  Jaime E. Forero-Romero$^{1}$ and
  Julian E. Mej\'ia-Restrepo$^{2,3}$ 
}

\affil{
$^{1}$ Departamento de F\'{i}sica, Universidad de los Andes, Cra. 1
No. 18A-10, Edificio Ip, Bogot\'a, Colombia \\
$^{2}$ Departamento de Astronom\'{i}a, Universidad de Chile, Camino el
Observatorio 1515, Santiago, Chile\\
$^{3}$ FACom-Instituto de F\'isica-FCEN, Universidad de Antioquia, Calle 70 No. 52-21, Medell\'in, Colombia
} 



\begin{abstract}
%
  We study the impact of cosmic variance in constraining the mass and
  occupation fraction of dark matter halos hosting \ly Emitting
  Galaxies at high redshift. We use an N-body simulation to construct
  mock fields with the same typical size of observed fields at
  $z=3.1$ to match the observed number density distribution across
  fields and the angular correlation function. In our model a dark
  matter halo with mass in the range $M_{\rm min}<M_{\rm h}<M_{\rm
    max}$ can only host one detectable LAE with a probability $0.1
  \leq f_{\rm occ}\leq 1.0$.  Our analysis shows that the clustering
  and number density information are insufficient to impose a tight
  constraint on the occupation fraction. On the other hand, the
  minimum mass and maximum mass can be constrained to the range
  $M_{\rm  max}<10^{12}$\hMsun\ and $10^{10.2}\hMsun\leq M_{\rm
    min}\leq 10^{11.5}\hMsun$. Additionally, we find that the
  consistent models have a narrow mass range, $\Delta M \equiv \log_{10}M_{\rm max} -
  \log_{10} M_{\rm min}$, smaller than $1.0$ dex, with a majority in
  the range $\Delta M<0.5$ dex. These results fall into constraints
  already presented in the literature. However, the wide range of
  values for the occupation fraction and the narrow mass range $\Delta
  M$ are novel results allowed by takinf into account the influence of
  cosmic variance on the statistics derived from observations.  
\end{abstract}


\keywords{cosmology: theory – cosmology: large-scale structure of universe –
  galaxies: formation – galaxies: high-redshift – galaxies: statistics
  – galaxy: haloes}



\section{Introduction}

Lyman-$\alpha$ emitting galaxies (LAEs) are helpful in a diverse range
of subjects in extragalactic astronomy. LAEs can be
used as probes of reionization \citep{Dijkstra11}, tracers of large
scale structure \citep{Koehler2007},  signposts for low metallicity
stellar populations, markers of the galaxy formation process at high
redshift \citep{Dayal2009,ForeroRomero2012} and tracers of active star
formation \citep{Guaita2013}. 

Capitalizing LAEs observations requires an understanding of
their place in the context of a given structure formation model in a
cosmological context. Under the current paradigm the dominant matter
content of the Universe is to be found in dark matter (DM) and each
galaxy is thought to be hosted by larger dark matter structure known
as a halo. \citep{Peebles1980,SpringelNature05}. 

Galaxy formation models find that halo mass predicts with high
accuracy galactic properties such as stellar mass and star formation
rate \citep{Behroozi2013a}. This suggests that the
physical processes that regulate the star formation cycle are 
dependent on halo mass.  For that reason, finding the typical dark
matter halo mass hosting LAEs represents an advance to understand the
nature of this galaxy population in the context of Lambda Cold Dark
Matter ($\Lambda$CDM) paradigm.  

Some theoretical attempts to solve this problem using an  ab-initio
approach. They start from the DM halo population to infer the
intrinsic star formation rates and \ly a luminosities. From these
values they estimate the amount of \ly photons that
escape each galaxy and compute the observed luminosity for each
galaxy. These models can predict different observables including: the
luminosity function, correlation function and the equivalent width
distributions. Such modelling has been implemented in semi-analytic
models \citep{Garel2012,Orsi2012} and  full N-body
hydrodynamical simulations \citep{Laursen2007, Dayal2009,
  ForeroRomero2011, Yajima2012}. 

However, these calculations involve many uncertain steps such as 
the estimation of the escape fraction, that is the fraction of \ly
photons that escape the galaxy to the observer. Given the resonant
nature of the \ly line, the escape fraction is sensitive to  the dust
contents, density, temperature, topology and kinematics of the neutral
Hydrogen in the interstellar medium (ISM). Solving the radiative
transfer of \ly photons in the ISM requires Monte Carlo
simulations. The process of finding a consensus on the expected value
for the \ly escape fraction in high redshift galaxies is still matter
of ongoing research
\citep{Neufeld1991,Verhamme2006,ForeroRomero2011,Dijkstra2012,Laursen2013,Orsi2012}.   

A different approach to infer the typical mass of halos hosting
LAEs is based on the spatial clustering information. This approach uses the fact
that in CDM cosmologies the spatial clustering of galaxies on large
scales is entirely dictated by the halo distribution
\citep{Colberg00}, which in turn has a strong dependence on halo
mass. Using measurements of the angular correlation function of LAEs,
observers have put constraints on the typical mass and occupation
fraction of the putative halos hosting these galaxies
\citep{Hayashino2004,Gawiser07,Nilsson2007,Ouchi2010}. In these
studies the observations are done on fields of $\sim 1$ deg$^{2}$ and
the conclusions derived on the halo host mass do not delve too deeply
into the possible impact of cosmic variance.  

Recently \cite{Yamada2012} observed a wide area of $2.4$ deg$^{2}$
under homogeneous instrumental and data reduction conditions. This data
set is constructed from 12 different sub-fields that allows us to use
clustering statistics and the cosmic variance among fields to
constrain the mass and occupation fraction of halos hosting LAEs. 


In this paper we investigate the impact of cosmic variance in
constraining the mass and occupation fraction of halos hosting LAEs.
We do this by constructing mock catalogs from cosmological simulations.
Our method populates each halo in the simulation with a LAE without
predicting a \ly  luminosity. This bypasses all the physical
uncertainties associated to star formation and radiative transfer.
We build mock surveys following the geometry of the observed fields by
\cite{Yamada2012} to compared them  against observations in terms of
the number density distribution and the angular correlation
function. This allows us to find a range of parameters of our model
that are consistent with observations while taking into account
cosmic variance.
 

This \documentname is structured as follows. In the next section we present
the simulation and the model used to produce the mock catalogs. We
also list the criteria used to compare the mocks against
observations. In \S \ref{sec:results} we present the main results for
the halo mass and occupation fraction. We continue in \S
\ref{sec:discussion} with a discussion under the light of other
constraints already presented in the literature and theoretical
considerations of galaxy formation studies. Finally, we present our
conclusions in \S \ref{sec:conclusions}.   

Throughout this \documentname we assume a $\Lambda$CDM cosmology with the
following values for the cosmological parameters, $\Omega_{m}=0.27$,
$\Omega_{\Lambda}=0.73$ and $h=0.70$, corresponding to the matter
density, vacuum density and the Hubble constant in units of 100 km
s$^{-1}$ Mpc$^{-1}$. 

\section{Methodology}

Our method is based on the comparison of observations and mock
catalogs that take into account cosmic variance. Our benchmarks are the
distribution of the surface number density across fields and the
angular correlation function.

In the next subsections we describe in detail the four key
elements int our work-flow. First, we present the 
observations we take as a benchmark. Second, we describe the main
characteristics of the N-body simulation and the halo catalogs we
use. Third, we list the important parameters of the simplified
model that we use to populate the halo catalogs with LAEs. Finally, we
describe the statistical tests we adopt to compare observations and mocks.

\subsection{Observational constraints}


\begin{figure}
\begin{center}
\includegraphics[width=0.95\linewidth,angle=0]{Fig1b.pdf}
\caption{ \label{fig:number_density} Cumulative distribution of LAE number
  densities in all the fields observed by \citet{Yamada2012}. The
  point with the highest surface density corresponds to the densest
  sub-field in the SSA22 field.}
\end{center} 
\end{figure}

The first observational constraint we use in this paper is the LAE number
density information at $z=3.1$ obtained by \cite{Yamada2012} from a survey
conducted with the Subaru 8.2m telescope and the Subaru Prime Focus Camera.
The camera has a field of view covering $34\times 27$ arc-min,
corresponding approximately to a comoving scale of $46\times35$ Mpc
$h^{-1}$ at $z=3.09$.  

The narrow band filter used in the survey is
centered at $4977$ \AA~with  $77$ \AA~width, corresponding to the
redshift range $z=3.062$-$3.125$ and $41$ \hMpc comoving scale for the
detection of the Lyman-$\alpha$ line centered at $z=3.09$. The authors
reported a total  $2161$  LAEs with an observed equivalent width, in
the observer frame, larger than $190$ \AA~over a total survey area of
$2.42$ deg$^{2}$ that includes 12 sub-fields,  this corresponds to an
average surface number density of $0.24\pm 0.01$ arcmin$^{-2}$.  The
distribution of the surface number densities is shown in Figure
\ref{fig:number_density}. 

The survey covered four independent fields:

\begin{enumerate}
\item The SSA22
field of $1.38$ deg$^2$ with $1394$ detected LAEs. This large field is
composed by 7 sub-fields. 

\item The Subaru/{\it
  XMM-Newton} Deep Survey (SXDS)-North, -Center and -South fields, with a
total of $0.58$ deg$^2$ and $386$ LAEs (3 sub-fields).

\item The Subaru Deep Field (SDF) with $0.22$ deg$^2$ and
$196$ LAEs (1 sub-field).

\item The Great Observatory Optical Deep Survey  (GOODS-N) field with
  $0.24$ deg$^2$ and $185$ LAEs (1 sub-field).   
\end{enumerate}

The surface number density distribution for these fields is shown in
Figure \ref{fig:number_density}.

There is abundant observational work done on LAEs at redshift $z=3.1$
\citep{Kudritzki2000,Matsuda2005,Gawiser2007,Nilsson2007,Ouchi2008}.
However, we use the data from \cite{Yamada2012} because
it has the largest covered area with homogeneous instrumentation
conditions (telescope, narrow band filter), data reduction pipeline
and conditions to construct the LAE catalog. This ensures that the
number density variations among fields are not due to different
observational conditions or criteria to construct the catalogs.

The second benchmark is the angular correlation function
(ACF). Unfortunately, \cite{Yamada2012} do  not report an ACF measurement for
their fields. Instead, we use the results by   \cite{Ouchi2008} who
reported the ACF is over a region of $1$ deg$^2$ over the SXDS
field.  

There are some differences between \cite{Ouchi2008} observations
and those by \cite{Yamada2012}. The details in the color selection,
corresponding limiting luminosities and EW thresholds are different
in these references. In the case of
\cite{Ouchi2008} the number density is $0.099\pm0.005$ arcmin$^{-2}$, which
is 50\% lower that the median value of $0.20$ arcmin$^{-2}$ for the
the general fields in \cite{Yamada2012}. In spite of these
differences \cite{Ouchi2008} provides the most similar conditions
to the observations presented by \cite{Yamada2012}.


We note that the field SSA22-1 has been known to harbor a significant
galaxy overdensity. \cite{Yamada2012} estimates that the densest
sub-field is likely to be a rare density peak with $3-4\sigma$
significance on the scale of $\sim 60$\hMpc.  All the other sub-fields
in SSA22 are average with similar number density as other blank fields. 

From Figure \ref{fig:number_density}  it is evident that there is only
one sub-field  that stands out as an outlier in the distribution,
corresponding to SSA22-1. All the other fields form a continous
distribution in number density. Therefore, using the whole SSA22
region as a benchmark in the number density distribution does not
impose any bias. The statistical test we use o compare mock
distributions against observations does not require a perfect match
with observations, i.e. the presence of a mock fields as dense as
SSA22-1 (which is very unlikely) is not required to consider that we
have a good match with observations.

\subsection{Simulation and halo catalogs}

We use results form the Bolshoi simulation \citep{Bolshoi} which
was performed in a cubic volume of 250 $h^{-1}$ Mpc comoving on a side. The
dark matter distribution is sampled using $2048^{3}$ particles. The
cosmological parameters are consistent with a Wilkinson Microwave
Anisotropy Probe (WMAP) ninth year data with a matter density
$\Omega_{\rm m} = 0.27$, cosmological constant
$\Omega_{\Lambda}=0.73$, dimensionless Hubble constant $h=0.70$, slope
of the power spectrum $n=0.95$ and normalization of the power
spectrum$\sigma_{8}=0.82$ \citep{hinshaw_etal13}.   This
translates into a particle mass of $m_{\rm p}=1.35\times 10^{8}$
$h^{-1}$ M$_{\odot}$.  

We use halo catalogs constructed with a Friend-of-Friends (FOF)
algorithm with a linking length of 0.17 times the inter-particle
distance. The catalogs were obtained from the publicly available
Multidark database \footnote{{\tt
    http://www.multidark.org/MultiDark/}} \citep{MultiDark}. For each
halo in the box we store its comoving position in the box (3-D
coordinates) and FOF mass. We focus our work on halos more massive
than $1\times 10^{10}$\hMsun resolved with at least $70$ particles.
%the reasons for this choice are explained in the next sub-section. 


\subsection{A model to populate halos with LAEs}
\label{subsec:mocks}


We assume that a dark matter halo can only host one
detectable LAE at most.  This is consistent with theoretical analysis
of the correlation function \citep{Jose2013b} and obsevations that
confirm a lack of class pairs in LAEs \cite{Bond2009}.

There are three parameters that decide whether a halo host a LAE: the
lower and upper bounds for the mass range ($M_{\rm min}< M_{\rm h} < M_{\rm max}$) 
and the fraction ($f_{\rm occ}$) of such halos that host a detectable
LAE. A physical interpretation of the occupation fraction $f_{\rm
  occ}$ convolves two phenomena: the actual presence of a star forming
galaxy in a halo and its detectability as a LAE.  


Our model does not assign a luminosity or escape fraction for each
LAE in order to maintain theoretical uncertainties to a minimum. With
this flexibility we can explore a wide range of possible masses for
the host halos without any strong theoretical prejudice regarding the
details of star formation and \ly escape fraction in high-redshift
galaxies.  


In what follows we note by the letter ${\mathcal M}$ a model
defined by a particular choice of the three scalar parameters $M_{\rm
  min}$, $M_{\rm  max}$ and $f_{\rm occ}$. For each model ${\mathcal
  M}$ we create a set of mock fields from disjoint volumes in the
simulation. Each volume has the same geometry probed by Suprime-CAM
and the narrow band filter, namely rectangular cuboids of dimensions
$46\times 35\times 41$ $h^{-3}$ Mpc$^{3}$ where the last dimension goes
in the redshift direction. This corresponds to a total area of $880$
arcmin$^{2}$ in each mock field. We construct a total $5\times 7
\times 6=210$ of such volumes from a snapshot in the Bolshoi
simulation. In each mock field a LAE is assigned to the position of a
dark matter halo if the halo mass is in the range allowed by the model
$M_{\rm min}<M_{\rm h}<M_{\rm max}$ and a random variable taken from
an homogeneous distribution $0\leq \xi<1$ is smaller than the occupation
fraction $\xi<f_{\rm occ}$.

Next we construct mock surveys by making groups of $12$ mock fields
out of the $210$ available volumes. In total $15$ mock surveys are
constructed for each model $\mathcal{M}$. The grouping of the $12$
mock fields into a mock catalog is done in two different ways. The
first is called {\texttt{match}} because it follows the clustering of the
observational fields. From the $12$ mock fields, $7$ are constructed
from contiguous fields in the simulation to mimic the SSA22 region,
$3$ are also contiguous between them but not to the first $7$ fields
to mimic the SXDS fields and finally $2$ non-contiguous fields to
imitate the SDF and GOODS-North field.   The second way to group the
mock fields is called {\texttt{random}}, whereby all the $12$ fields
are selected in such a way as to avoid that any two volumes are
contiguous. In this \documentname we only report the results obtained
by the {\texttt{match}} method since we do not find any significant
difference with respect to the {\texttt{random}} method.


Figure \ref{fig:distros} shows the spatial distribution for one mock
survey constructed using the {\texttt{match}} method. Each field
corresponds to one of the observational fields. The model parameters
to build the mock are $M_{\rm min}=10^{10.4}$\hMsun, $M_{\rm
  max}=10^{10.5}$\hMsun and $f_{\rm occ}=0.1$. The figure shows only
one out of the $15$ different mock surveys that are constructed for
each model. We note that we only use $15\times 12=180$ mock fields out of the
total of $210$ available sub-volumes. The reason is that the {\texttt{match}}
method imposes constraints on the way the $7$ fields mimicking the
SSA22 can be distributed. This restriction makes unable some of the
sub-volumes in the box. We decide to keep the number of mock surveys
fixed to $15$ also for the {\texttt{random}} method in order to allow a
fair comparison between the two methods.

\subsection{Exploring and selecting good models}

We make a thorough exploration of the parameter space for the models
${\mathcal M}$ where $\log_{10} M_{\rm min}$ takes $30$ values from $10.0$ up
to $12.9$ with an even spacing of $0.1$ dex, $\log_{10} M_{\rm max}$
takes values in the same range as $\log_{10}M_{\rm min}$ with a
displacement of $0.1$ dex in the whole range. The occupation fraction
$f_{\rm occ}$ takes $10$ different values from $0.1$ to $1$ regularly
spaced by $0.1$. In total the number of different models ${\mathcal
  M}$ that are explored is $30 \times 30 \times 10 = 9000$.  

The lower limit for the parameter $M_{\rm min}$ is set by the minimum
occupation fraction we decide to consider. At $M_{\rm
  min}=10^{10}$\hMsun the halo number density around that mass range
is $\sim 10$ times higher than the observational constraints
for LAEs. This means that models in that mass range and an occupation fraction $f_{\rm  occ}=0.1$ have the possibility to be compatible with
observations. Lower values for $M_{\rm min}$ require $f_{\rm
  occ}<0.1$, which are not considered in this \documentname.  In
  turn exploring occupation fractions on the order of $f_{\rm occ}=0.01$ 
  only makes sense for halo populations in the mass range of
  $10^{9}$\hMsun which are abundant enough to fit observations with a
  very low occupation fraction. In our case, this is a mass range unresolved
  by the Bolshoi simulation and therefore cannot be considered in the
  present study.

For each mock survey generated in a given model ${\mathcal M}$ we
compute the surface density in its $12$ mock fields. We perform a
Kolmogorov-Smirnov (KS) to compare these values against the $12$
observational values. This tests gives us a value $0<P<1$ to
reject the null hypothesis, namely that two data sets come from the
same distribution. In this paper we consider that for values $P>0.05$
the two distributions can be thought as coming from the same
distribution.

We begin by considering that  a model ${\mathcal M}$ with at least one
mock survey (out of 15) consistent with observations has viable
parameters to host LAEs. From that we consider a stronger constraints
to reduce the number of models by asking that all the 15 mocks  to be
consistent with observations and analyze again the properties of the
resulting models. Finally we add the ACF as an additional constraint
and consider all the models having their $15$ mocks consistent with
observations. 

The ACF is computed using  the Landy \&  Szalay estimator
  \citep{Landy1993}  on fields of size $1$ deg$^2$ to be compared
  against the results reported by \cite{Ouchi2010}.

The observed and mock ACF are fit to a power-law function:
\begin{equation}
\omega(\theta) = \left(\frac{\theta}{\theta_{0}}\right)^{-\beta}, 
\label{eq:fitting}
\end{equation}
where $\theta_0$ and $\beta$ are free parameters. The fit is done
using a least square minimization procedure. For each mock field we
obtain a covariance matrix that gives us the uncertainty in the
parameters $\beta$ and $\theta_0$.  We consider that a mock field is
consistent with observations if the two parameters $\beta$ and
$\theta_0$ are equal within a $1$-$\sigma$ range. We perform this
comparison only with the fields constructed using the {\texttt{match}}
method.
 
\section{Results}
\label{sec:results}



\begin{figure}
\begin{center}
\includegraphics[width=0.95\linewidth,angle=0]{Fig1.pdf}
\caption{ \label{fig:halos} Surface density of dark 
  matter halos as a function of a minimum halo mass. Each line
  represents one of the $210$ volumes of dimensions $46\times 35\times
  41$ $h^{-3}$ Mpc$^{3}$ in the Bolshoi simulation. The horizontal
  gray band represents the range of surface densities observed for
  LAEs at $z=3.1$ as reported by \citet{Yamada2012} and the dashed
  line the observational results by \citet{Ouchi2008}.}
\end{center} 
\end{figure}


\begin{figure*}
\begin{center}
\includegraphics[width=0.46\linewidth,angle=0]{Fig7_corr_params.pdf}  
\includegraphics[width=0.46\linewidth,angle=0]{Fig7_mmin_vs_theta.pdf} 
\end{center}
\caption{Left panel. Values for the free parameters $\theta_{0}$ and $\beta$
in the fitting formula (Eq. \ref{eq:fitting}) for the angular
correlation function. Blue dots correspond to simulations and the
green line to observations by \citet{Ouchi2008,Ouchi2010}. The error
bars in the theoretical data correspond to the quadratic average of
the fitting uncertainties for each mock survey. Right panel. Values of
$\theta_{0}$  and $\log_{10}(M_{\rm min}/M_{\odot} h^{-1})$. Blue and
green point represents the models where  $\log_{10}(M_{\rm max}/M_{\odot}
h^{-1})<12.0$ and  $\log_{10}(M_{\rm min}/M_{\odot} h^{-1})>12.0$
respectively. .  The horizontal gray band represents the observational
constraints on $theta_{0}$ established by
\citet{Ouchi2008,Ouchi2010}. Green points has been displaced by
$0.01$ dex of its original value to avoid overlapping between blue and
green points.} \label{fig:correlation_parameters}   
\end{figure*}


The main purpose of this section is to show how the different
observational constraints narrow down the parameters space of allowed
models. Each sub-section presents the effect of adding a new piece of 
observational or statistical evidence. 


\subsection{Dark Matter Halo Number Density}

The right panel in Figure \ref{fig:halos} shows the  integrated dark matter halo surface
density as a function of  minimum halo mass $M_{\rm min}$. Each line
corresponds to one of the 210 sub-volumes in the Bolshoi
simulation. The gray band indicates the surface density values for
LAEs allowed by observations \citep{Yamada2012}. The dashed lines
represent the average values in the fields observed by
\citep{Ouchi2008}. 
 
This plot allows us to understand why only a specific range of
models ${\mathcal M}$ can be expected to be consistent with
observations. From Figure \ref{fig:halos} we can read that models with
a minimum mass $M_{\rm min}>10^{12}$\hMsun always have a
surface number density lower than the observational constrain, making
them incompatible with observations; there are simply too few halos
compared to observed LAEs. The opposite is true in models with $M_{\rm
  min}<10^{11.0}$\hMsun, which have a surface number density larger
observations. In those cases the maximum mass $M_{\rm max}$ and the
occupation fraction $f_{\rm occ}<1.0$  can be tuned in order to lower
the halo number density to match observations.   

Figure \ref{fig:halos} also illustrates the impact of cosmic
variance. At fixed minimum mass there is an scatter of $0.3-0.6$ dex
in the number density abundance, which is of the same order of
magnitude as the scatter in the observational data.  As a consequence,
the variation in the number density in mocks for models with the same
mass range and occupation fraction can be by factors of $\sim 2-5$.
This scatter induced by cosmic variance is naturally included in the
mock construction process. 



\subsection{Maximally Consistent Models}



We impose two major constraints in our models. First, we require a 
models to have all its 15 mocks with a number density distribution
consistent with observations. Second, we require that the average
values among all mocks for the $\theta_{0}$-$\beta$ values in the
correlation function fit to be consistent with observations
\citep{Ouchi2008}. 

The left panel in Figure \ref{fig:correlation_parameters} shows the
results for the best estimates of $\theta_{0}$-$\beta$  used in the
ACF parameterization on the best models previously selected.  Blue
circles with error bars represent the results from the mocks and the
horizontal line the observational results of \cite{Ouchi2008} over a
field with average number density where the $\beta$ parameter was
fixed in the fit. From this Figure we observe that there are models
with too large values of $\theta_0$ and/or $\beta$ that can be ruled
out as inconsistent.


Figure \ref{fig:restriction_mock_and_f_occ_corr} presents the
consistent models in the planes $M_{\rm min}$-$M_{\rm
  max}$ and $M_{\rm   min}$-$f_{\rm occ}$. A model is consistent with
observations if there is a $1$-$\sigma$ overlap between both the
correlation length $\theta_0$ and the exponent $\beta$.  We find that
we end up with $40$ models consistent with the ACF constraints.  This
is a reduction of a factor of $\sim 2$ with respect to the number of
models consistent with the number density distribution.



\section{Discussion}
\label{sec:discussion}




Out of the initial set of $9000$ models we end up with $40$ that are
consistent with the observational constraints. To facilitate the
discussion of these models we define a new quantity, the halo mass
range $\Delta M\equiv \log_{10}M_{\rm max} - \log_{10}M_{\rm  min}$, which
together with the occupation fraction, $f_{\rm occ}$, and the minimum
mass $M_{\rm min}$ allows us to classify all the successful models into
three families:     
  

\begin{itemize}
\item[(1)] Low occupation fraction $f_{\rm occ}\leq 0.2$ and narrow
  mass range $\Delta M\leq 1.0$ 
  dex: 16 models. 
\item[(2)] High occupation fraction $f_{\rm occ}> 0.2$ and
  narrow mass range $\Delta M\leq 1.0$: 17 models 
\item[(3)] Low occupation fraction $f_{\rm occ}\leq 0.2$ 
  and wide mass range $\Delta M>1.0$: 7 models
\end{itemize}


A complete parameter list for the three families is presented in Tables
\ref{table:firstfamily}, \ref{table:secondfamily}  and 
\ref{table:thirdfamily}.  

The models in the third family are barely consistent with the
  constraints from the ACF. These models they have another particular
  feature: their minimum mass is exactly $M_{\rm min}=10^{10.9}\hMsun$
  and $M_{\rm max}>10^{12.0}\hMsun$ . They have values of $\beta$
  close to $1.0$ (instead of the observational value of $0.8$) and
  $\theta_0$ on the range of $13^{\prime}$ (the mean observational
  value is close to $8^{\prime}$) but are considered consistent
  because their large error bars barely overlap with the observational
  expectation (see figure \ref{fig:correlation_parameters}).
  Because these models are are barely compatible with observations and
  are a minority of the consistent models, we exclude them from the
  discussion.

\begin{figure*}
\begin{center}
\includegraphics[width=0.45\linewidth,angle=0]{Fig6_mass.pdf}
\hspace{5mm}
\includegraphics[width=0.45\linewidth,angle=0]{Fig6_f_occ.pdf}
\end{center}
\caption{$M_{\rm min}$-$M_{\rm max}$ (left) and $M_{\rm
    min}$-$f_{\rm occ}$ (right) planes showing the models fulfilling both
   constraints on the maximal number of consistent mocks and the
  angular correlation function from \citet{Ouchi2008,Ouchi2010}.   
  \label{fig:restriction_mock_and_f_occ_corr}} 
\end{figure*} 


\subsection{Two relevant features in the preferred models}
There are two interesting features in the remaining two first
families. First, the occupation fraction can take any value from $0.1$
to $1.0$. Second, the halos hosting LAEs all have a narrow mass range
$\Delta M\leq 1.0$, half of them $\Delta M\leq 0.5$.


The existence of models with high occupation fractions $f_{\rm
    occ}>0.2$ is unexpected from previous clustering analysis that do
  not take  account the effect of cosmic variance in the same way we
  propose in this \documentname. Nevertheless, values of an occupation
  fraction $f_{\rm occ}=0.1$ and the mass range
  predicted by all of our models is consistent with already published
  observational results \citep{Gawiser07,Ouchi2010}. This suggests
  that explicitly taking cosmic variance of the observed fields into
  account (the only point not included in the analysis already cited)
  hinders the possibility of constraining the LAEs occupation
  fraction.

Concerning the narrow mass range $\Delta M
  <1.0$ the question arises of how can we give a physical
  interpretation for the existence of such  a narrow mass range. We
  can start with a reasonable   assumption. Namely, that star formation
  rate increases with halo's  mass. Under this assumption the cut at
  the low mass end, $M_{\rm min}$, can be interpreted in terms of the
  minimal star formation rate required to produce a \ly luminosity
  above observational detection thresholds.   A cut at higher halo
  masses $M_{\rm  max}$ requires a different justification. There are
  two complementary physical scenarios that could provide it.

One scenario can be presented in terms of a decreasing escape fraction
of \ly radiation in massive systems. Detailed galaxy formation models
support the idea that massive galaxies with higher metallicities have
larger dust contents and a less concentrated ISM than lower mass
systems. Due to the resonant nature of the \ly line the probability of
absorption  of \ly photons increases in massive systems, producing
high absorption of the \ly line but not of UV continuum or other
non-resonant lines \citep{Laursen2009,ForeroRomero2011}. In a second
scenario larger systems have more extended gaseous envelopes which due
to resonance effects of the \ly line, induces a low surface brightness
and a broader line, making these systems less observable in narrow
band filter surveys \citep{Laursen2009,Zheng2010}.    


\subsection{Comparison to previous clustering estimates}

Observational evidence based on the ACF inferred from photometric
measurements in the Extended Chandra Deep Field South has shown that
the median dark matter masses of halos hosting LAEs is
$\log_{10}M_{\rm  med}=10.9^{+0.5}_{-0.9}$\Msun, with a corresponding
occupation fraction of $1-10\%$  \citep{Gawiser07}.  \cite{Ouchi2010}
presents analysis of LAE observations in the redshift interval
$3.1<z<7.0$ and at $z=3.1$ They quote an average mass for the host
dark matter halos of $M_{h}=2.9^{+24.0}_{-2.9}\times 10^{10}$ \hMsun
with a corresponding duty cycle of $0.008\pm 0.03$.  

Our results are in a general good agreement with those estimates for
the host mass and the upper limit of the occupation fraction. This is
not completely unexpected given that we have also required consistency
with ACF measurements. These expectations are mostly matched by
the first family of models, also summarized in Table
\ref{table:firstfamily}. These models, which favor only the  low
($\sim 10\%$) occupations fractions, are also consistent in that
regard with the observational expectations by \cite{Gawiser07}. 

The novelty in our results is that we have a detailed estimate for 
host halo mass range together with the occupation fraction. This allows us
to show that the halo mass range can, in some cases, be narrow $\Delta M <
0.3$ dex, something that cannot be inferred from ACF analysis alone.
Furthermore, In contrast to \cite{Gawiser07} and \cite{Ouchi2010} we
find that an ACF analysis is not enough to rule
out models with a high occupation fraction $f_{\rm occ}>0.2$.




\begin{table}
\begin{center}
\begin{tabular}{cccc}\hline\hline
$\log_{10}M_{\rm min}$ & $\log_{10}M_{\rm max}$ & $f_{\rm occ}$ & $\Delta M$\\\hline
 10.4 &10.7 & 0.1& 0.3 \\
 10.5 &10.9 & 0.1& 0.4 \\
 10.5 &11.0 & 0.1& 0.5 \\
 10.6 &10.8 & 0.2& 0.2 \\
 10.6 &11.2 & 0.1& 0.6 \\
 10.6 &11.3 & 0.1& 0.7 \\
 10.6 &11.4 & 0.1& 0.8 \\
 10.6 &11.5 & 0.1& 0.9 \\
 10.6 &11.6 & 0.1& 1 \\
 10.7 &11.0 & 0.2& 0.3 \\
 10.8 &11.2 & 0.2& 0.4 \\
 10.8 &11.3 & 0.2& 0.5 \\
 10.8 &11.4 & 0.2& 0.6 \\
 10.9 &11.9 & 0.2& 1.0 \\
 10.9 &11.7 & 0.2& 0.8 \\
 10.9 &11.8 & 0.2& 0.9 \\\hline
\end{tabular}
\end{center}
\caption{\label{table:firstfamily}
List of parameters for the first
  family of models. Narrow mass range $\Delta M\leq 1.0$ dex and low
  occupation fraction $f_{\rm occ}\leq 0.3$.} 
\end{table}


\begin{table}
\begin{center}
\begin{tabular}{cccc}\hline\hline
$\log_{10}M_{\rm min}$ & $\log_{10}M_{\rm max}$ & $f_{\rm occ}$ & $\Delta M$\\\hline
 10.5 &10.6 & 0.3 & 0.1 \\
 10.6 &10.7 & 0.4 & 0.1 \\
 10.7 &10.8 & 0.5 & 0.1 \\
 10.8 &10.9 & 0.6 & 0.1 \\
 10.8 &11.0 & 0.4 & 0.2 \\
 10.9 &11.0 & 0.8 & 0.1 \\
 10.9 &11.0 & 0.9 & 0.1 \\
 10.9 &11.1 & 0.5 & 0.2 \\
 10.9 &11.3 & 0.3 & 0.4 \\
 11.0 &11.1 & 1.0 & 0.1 \\
 11.0 &11.2 & 0.6 & 0.2 \\
 11.0 &11.4 & 0.4 & 0.4 \\
 11.1 &11.3 & 0.7 & 0.2 \\
 11.1 &11.3 & 0.8 & 0.2 \\
 11.1 &11.4 & 0.6 & 0.3 \\
 11.2 &11.4 & 0.9 & 0.2 \\
 11.2 &11.4 & 1.0 & 0.2 \\\hline
\end{tabular}
\end{center}
\caption{\label{table:secondfamily}List of parameters for the second
  family of models. Narrow mass range $\Delta M\leq 1.0 $ dex and high occupation fraction $f_{\rm occ}>0.2$.}
\end{table}


\begin{table}
\begin{center}
\begin{tabular}{cccc}\hline\hline
$\log_{10}M_{\rm min}$ & $\log_{10}M_{\rm max}$ & $f_{\rm occ}$ & $\Delta M$\\\hline
 10.6 &11.7 & 0.1& 1.1 \\
 10.9 &12.0 & 0.2& 1.1 \\
 10.9 &12.1 & 0.2& 1.2 \\
 10.9 &12.2 & 0.2& 1.3 \\
 10.9 &12.3 & 0.2& 1.4 \\
 10.9 &12.6 & 0.2& 1.7 \\
 10.9 &12.8 & 0.2& 1.9 \\\hline
\end{tabular}
\end{center}
\caption{\label{table:thirdfamily}List of parameters for the third
  family of models. Wide mass range $\Delta M> 1.0$ dex and low
  occupation fraction $f_{\rm occ}\leq 0.2$. These models are barely
  consistent with the constraints from the Angular Correlation
  Function and have been excluded from the main discussion.}
\end{table}


\subsection{In the context of abundance matching models}

The abundance matching methods are based on observational results for
Lyman Break Galaxies (LBGs) \citep{Behroozi2013a,Behroozi2013b}.  In
the case of \cite{Behroozi2013a} the minimum halo mass considered to
be relevant in their analysis is $10^{11.4}$\hMsun. They report
stellar mass mass around $(1.0\pm0.3)\times 10^{9.0}$ \hMsun, while
their star formation rate is in the range $0.6\pm 0.2$ \Msun yr$^{-1}$,
which nevertheless are close to the lower bound of values inferred for
LAEs at high redshift \citep{Gawiser2007,Nilsson2009,Pentericci2009}. 

In our results, all the preferred models have a halo mass range lower
than the minimum of $M_{\rm min}<10^{11.4}$\hMsun considered in
abundance matching at $z=3$. Our results confirm the expectations
that most of  LAEs are to be found in less massive halos LBG hosts. A
detailed analysis of the spectral and photometric properties of LAEs
coupled to the kind of analysis performed in this paper can be a guide
in the study of the properties of low mass dark matter halos at
$z=3.1$, extending the capabilities of abundance matching methods.

\subsection{Caveats of our method}

There are three minor caveats for the work presented here that are important
to keep in mind. 

The first caveat is the precise values for the mass intervals. These
values are quoted from halos defined using a FOF halo
finder. Different halo finders and definitions for the detection
density threshold can yield different masses up to a factor $\sim 2$\citep{More2011}. For 
instance a Friends-of-Friends algorithm with linking length $l=0.20$
times the average inter-particle distance finds halos on average $1.4$
times less massive than halos defined  with an spherical overdensity
algorithm halos \citep{Bolshoi}. Therefore, the mass values for
$M_{\rm min}$ and $M_{\rm max}$ should not be considered exact within
less than $\sim 0.2$ dex. However, this represents a systematic change
in the mass values and the mass interval $\Delta M$ quoted here should
remain unchanged.

The second caveat is the exact value that was used to turn
  filter values into cosmological volumes. Strictly speaking the
  volume probed is a function of the \ly luminosity. Brighter LAEs
  probe a larger volume than fainter ones \citep{Gronwall07}. A proper
way to account for this effect would require to assign luminosities to
each mock LAE and process them in a similar way as mock
observations. However, we can estimate that under the considerations
presented in \citep{Gronwall07}, one could expect the survey depth to
change at most by a factor of $\sim2$. Considering the dependence
of the number density on halo mass (right panel of Figure 1), such
change in volume could be compensated by different values for the
favored model parameters $M_{\rm  min}$ and $M_{\rm max}$ by about the
same factor of $\sim2$ or $0.3$ dex. 

A third caveat is the comparison of results for the number density
  statistics and the ACF from surveys with different EW cuts. In this
  sense we highlight again that the two filters and selection
  criteria used by \citep{Ouchi2008} and \citep{Yamada2012} are only
  slightly different. The median number density in the general fields by
  \citep{Yamada2012} is $0.20$ arcmin$^{-2}$, while in
  \citep{Ouchi2008} is $0.099\pm0.005$ arcmin$^{-2}$. This gives us
  confidence that the two different samples can give a self-consistent
  picture.




\subsection{On the reproducibility of our results}

All the software, raw and processed data to produce the results
and plots in this paper are publicly available in a github
repository \footnote{https://github.com/forero/CosmicVarianceLAES}. Most
of the code to produce the plots can be found as an Ipython notebook
\citep{IPython} in the same repository.



\section{Conclusions}
\label{sec:conclusions}

In this \documentname we quantify the impact of cosmic variance in
constraining the  mass and occupation fraction of dark matter halos
hosting Lyman $\alpha$ Emitters at redshift $z=3.1$ in a $\Lambda$CDM
cosmology.  To this end we build a large number of mock catalogs
matching observational geometries. The mocks are constructed from a
N-body simulation by assigning a single LAE to a DM halo.
 
We then proceed with a thorough exploration of the space of free
parameters $M_{\rm min}, M_{\rm max}$ and $f_{\rm occ}$ describing the
model. We look for consistency with two observational constraints: the
surface number density distribution and the angular correlation
function. Out of the initial $9000$  combinations of parameters in the
model we find $40$ combination of parameters consistent with
observations. 

We find two interesting features in these models. First, the
occupation fraction remains unconstrained; second, many models have a
narrow mass range models with $\Delta M \leq 0.5$.  

The wide range of occupation fraction values is unexpected from
previous studies \citep[i.e.][]{Gawiser2007,Ouchi2010}  that
unanimously favor low values on the range of $10^{-2}$. Nevertheless,
all the halo mass ranges deduced in previous analysis are a subset of
the models we find in this paper, all of the best models support the
notion that the most massive halos at $z=3.1$ do not host detectable
LAEs.   

  From the point of view of Dark Energy surveys such as the Hobby-Eberly
  Telescope Dark Enery Experiment (HETDEX) the
  narrow mass range for LAEs can be seen as an advantage because its
  simplicity in their modeling aiming at recovering cosmological
  parameters We foresee that the observations with new instruments
(such as the Multi Unit Spectroscopic Explorer (MUSE), Hyper
SuprimeCam and HETDEX) covering larger fields and a wider range of
luminosities will be key in reducing cosmic variance and imposing
tighter constraints on the properties of dark matter halos hosting
LAEs. 

Finally, additional modeling for \ly radiation transfer is needed to
put tighter constraints on  occupation fraction in high redshift
halos, also paying attention to other physical phenomena, such as the
stochasticity \citep{ForeroRomero2013} in the star formation process,
which might play a role in introducting detection biases in high redshift
LAEs.


\section*{Acknowledgments} 

J.E.F-R was supported by the FAPA grant by Vicerrector\'ia de
Investigaciones at Universidad de los Andes.  

J.E.F-R thanks the hospitality of Changbom Park and the Korea
Institute for Advanced Study where the first full draft of this paper
was completed. The authors also thank Peter Laursen, Paulina Lira, 
Alvaro Orsi and Mark Dijkstra for helpful comments on the physical
interpretation and presentation of our results. 


The MultiDark Database used in this paper and the web application providing online access to it were constructed as part of the
activities of the German Astrophysical Virtual Observatory as result
of a collaboration between the Leibniz-Institute for Astrophysics
Potsdam (AIP) and the Spanish MultiDark Consolider Project
CSD2009-00064. The Bolshoi and MultiDark simulations were run on the
NASA's Pleiades supercomputer at the NASA Ames Research Center.



\bibliographystyle{apj}
\bibliography{references}

\end{document}

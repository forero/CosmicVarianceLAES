\documentclass[usenatbib]{mn2e} 
\usepackage{amsmath} 
\usepackage{amssymb} 
\usepackage{graphics}
\usepackage{graphicx}
\usepackage{epsfig}  
\def\be{\begin{equation}}
\def\ee{\end{equation}}
\def\ba{\begin{eqnarray}}
\def\ea{\end{eqnarray}}

\newcommand{\documentname}{paper~}
\newcommand{\match}{{\tt match}~}
\newcommand{\apj}{ApJ}  
\newcommand{\apjs}{ApJS}  
\newcommand{\apjl}{ApJL}  
\newcommand{\aj}{AJ}  
\newcommand{\mnras}{MNRAS}  
\newcommand{\mnrassub}{MNRAS accepted}  
\newcommand{\aap}{A\&A}  
\newcommand{\aaps}{A\&AS}  
\newcommand{\araa}{ARA\&A}  
\newcommand{\nat}{Nature}  
\newcommand{\physrep}{PhR}
\newcommand{\pasp}{PASP}    
\newcommand{\pasj}{PASJ}    

\newcommand{\kms}{\,km~s$^{-1}$}
\def\squig{\sim\!\!}
\newcommand{\LCDM}{$\Lambda$CDM~}
\newcommand{\beq}{\begin{eqnarray}}  
\newcommand{\eeq}{\end{eqnarray}}  
\newcommand{\zz}{$z\sim 3$} 
\newcommand{\avg}[1]{\langle{#1}\rangle}  
\newcommand{\ly}{{\ifmmode{{\rm Ly}\alpha}\else{Ly$\alpha$}\fi}}
\newcommand{\hMpc}{{\ifmmode{h^{-1}{\rm Mpc}}\else{$h^{-1}$Mpc }\fi}}  
\newcommand{\hGpc}{{\ifmmode{h^{-1}{\rm Gpc}}\else{$h^{-1}$Gpc }\fi}}  
\newcommand{\hmpc}{{\ifmmode{h^{-1}{\rm Mpc}}\else{$h^{-1}$Mpc }\fi}}  
\newcommand{\hkpc}{{\ifmmode{h^{-1}{\rm kpc}}\else{$h^{-1}$kpc }\fi}}  
\newcommand{\hMsun}{{\ifmmode{h^{-1}{\rm {M_{\odot}}}}\else{$h^{-1}{\rm{M_{\odot}}}$}\fi}}  
\newcommand{\hmsun}{{\ifmmode{h^{-1}{\rm {M_{\odot}}}}\else{$h^{-1}{\rm{M_{\odot}}}$}\fi}}  
\newcommand{\Msun}{{\ifmmode{{\rm {M_{\odot}}}}\else{${\rm{M_{\odot}}}$}\fi}}  
\newcommand{\msun}{{\ifmmode{{\rm {M_{\odot}}}}\else{${\rm{M_{\odot}}}$}\fi}}  
\newcommand{\lya}{{Lyman$\alpha$~}}
\newcommand{\clara}{{\texttt{CLARA}}~}
\newcommand{\rand}{{\ifmmode{{\mathcal{R}}}\else{${\mathcal{R}}$ }\fi}}  
\newcommand{\Lsun}{\mbox{\,$L_{\odot}$}}
\newcommand{\like}{\mathscr{L}}
\newcommand{\bftheta}{\mathbf{\Theta}}
\newcommand{\degree}{\ensuremath{^\circ}}
\def\spose#1{\hbox to 0pt{#1\hss}}
\def\simlt{\mathrel{\spose{\lower 3pt\hbox{$\mathchar"218$}}
     \raise 2.0pt\hbox{$\mathchar"13C$}}}
\def\simgt{\mathrel{\spose{\lower 3pt\hbox{$\mathchar"218$}}
     \raise 2.0pt\hbox{$\mathchar"13E$}}}
\font\smcap=cmcsc10

\begin{document}

\title[Dark Matter Halo Mass for LAEs  at $z=3.1$]{Using cosmic
  variance to constrain the dark matter halo mass of Lyman-alpha
  emitting galaxies at $z$=3.1} 
  
\author[~J.~E. Forero-Romero and J. Mejia]{
\parbox[t]{\textwidth}{\raggedright 
Jaime E. Forero-Romero$^{1}$ and
Julian Mej\'ia$^{2}$ 
}
\vspace*{6pt}\\
$^{1}$ Departamento de F\'{i}sica, Universidad de los Andes, Cra. 1
No. 18A-10, Edificio Ip, Bogot\'a, Colombia \\
$^{2}$ Departamento de Astronom\'{i}a, Universidad de Chile, Camino el Observatorio 1515, Santiago, Chile}

\maketitle

\begin{abstract}
We use cosmological N-body dark matter only simulations to constrain the
characteristic mass of dark matter halos hosting Lyman-Alpha Emitting
(LAE) galaxies at a redshift of $z=3.1$. The method is based on
matching the statistics for the number density between mock and
observed fields. The mock fields are constructed using a simple model
where a dark matter halo can only host one LAE with a probability
$f_{\rm occ}$ if its mass is found withing a certain range mass range
delimited by two threshold values, $M_{\rm min}$ and $M_{\rm max}$. 
We also maximize the number of mocks surveys consistent with
observations and impose consistency with the angular correlation
function. Under these conditions we find that LAEs are preferentially
hosted by halos in very narrow mass ranges less than 0.5 dex in width,
with the minimum mass $10\leq \log_{10}M_{\rm min}\leq 10.9$ and the
occupation fraction $f_{\rm   occ}\leq 0.2$.  Our finding suggest that
the most massive dark matter halos at that epoch do not host the
brightest LAEs. This also gives support to observational evidence that
says that only a small fraction of star forming galaxies can be
actually detected as LAEs.
\end{abstract}

\begin{keywords}
{galaxies: kinematics and dynamics, Local Group, methods:numerical}
\end{keywords}


\section{Introduction}

Lyman-$\alpha$ emitting galaxies (LAEs) have become in the last decade a 
central topic in studies of structure formation in the Universe. They 
are helpful in a diverse range of fields. LAEs can
be used as probes of reionization \citep{Dijkstra11}, tracers of
large scale structure \citep{Koehler2007}, 
signposts for low metallicity stellar populations and markers of the
the galaxy formation process through cosmic history \citep{ForeroRomero2012}.  


At the same time, theoretical and observational developments have
contributed to the emergence of a paradigm to describe structure
formation in a cosmological context. In this context it is considered
that dominant matter content of the Universe is to be found in dark
matter, whereby each galaxy is hosted by larger dark matter structure
known as a halo. 

Most models of galaxy formation find that the mass of the halo can be
used to predict properties of the galaxy such as its stellar mass and
star formation rate \citep{Behroozi2012}. Processes that regulate the
star formation cycle are also though to be strongly dependent on its
mass. Furthermore, the spatial clustering of galaxies on large scales
is entirely dictated by the halo distribution.  For the reasons
mentioned above, finding the typical dark matter halo mass hosting
LAEs represents a significant step forward to understand the nature of
this population in the context of Lambda Cold Dark Matter
($\Lambda$CDM) paradigm.  

Some theoretical approaches to this problem have been based on a
forward modeling. Starting from the DM halo population, the
corresponding intrinsic star formation properties are infered and
statistics such as the luminosity function, the correlation function
and the equivalent width distributions. Such modelling has been
implemented from analytic considerations, semi-analytic models
 and 
full N-body hidrodynamical simulations
\citep{Dayal2009, ForeroRomero2011, Yajima2012, ForeroRomero2012} . 

Added to the uncertainties in the astrophysical processeses describing
star formation in galactic populations, a highly debated steps in this
approach is the calculation of the fraction of Lyman-$\alpha$ photons
that escape the galaxy to the observer. Given the resonance nature of
the line, the radiative transfer of Lyman-$\alpha$ is sensitive to the
density, temperature, topology and kinematics of the neutral Hydrogen
in the interstellar medium (ISM) \citep{Neufeld1991, ForeroRomero2011,
Laursen2013}.  

This complexity makes the use of monte-carlo simulations for the
radiative transfer a required tool to obtain physically sound results,
although the degeneracy in the physical parameters involved in the
problem makes it difficult to achieve a robust consensus on what is
the theoretical expected value for the Lyman-$alpha$ escape fraction
in high redshift. 


Throughout this \documentname we assume a $\Lambda$CDM cosmology with the
following values for the cosmological parameters, $\Omega_{m}=0.27$,
$\Omega_{\Lambda}=0.73$ and $h=0.70$, corresponding to the matter
density, vacuum density and the Hubble constant in units of 100 km
s$^{-1}$ Mpc$^{-1}$. 

\section{Methodology}

Our method to constrain the typical mass of a dark matter halos
hosting LAEs at $z=3.1$ is based on the comparison of observational
results on the surface number density and the predictions of a simple
model that uses the outputs from cosmological N-body simulations. 

In the next subsections we describe in detail the four key elements of
this workflow. First, we present the observations we take as a
benchmark. Second, the N-body simulation and the halo catalogs we
use. Third, the simplified model that allows us to translated halo
catalogs into mock LAE observations. Fourth, the statistics we use to
compare observational results against our theoretical predictions.

\subsection{Observational Constraints}

The observational benchamark we use in this paper is the LAE number
density information at $z=3.1$ obtained by the panoramic narrow-band
survey presented by \cite{Yamada2012} from a survey  conducted with
the Subaru 8.2m telescope and the Subaru Prime Focus Camera, which has
a field of view covering $34\times 27$ arcmin, corresponding to a
comoving scale of $46\times35$ Mpc $h^{-1}$ at $z=3.09$. The narrow
band filter is centered at $4977$ \AA with  $77$\AA width,
corresponding to the redshift range $z=3.062-3.125$ and $41$ Mpc
$h^{-1}$ comoving scale for the detection of the Lyman-$\alpha$ line
centered at $z=3.09$. The authors report a total  $2161$ LAEs with an
observed equivalent width larger than $190$\ over a total survey area
of $2.42$ deg$^{2}$, this corresponds to  averaga surface number
density of $0.20\pm 0.01$ arcmin$^{-2}$.   

The survey covered four independent fields. The first is the SSA22
field of $1.38$ deg$^2$ with $1394$ detected LAEs, this field has been
known to harbor a region with a large density excess of galaxies. The
second observed region is composed by the fields Subaru/{\it
  XMM-Newton} Deep Survey (SXDS)-North, -Center and -South, with a 
total of $0.58$ deg$^2$ and $386$ LAEs. The third and fourth fields
are the Subaru Deep Field (SDF) with $0.22$ deg$^2$ and $196$ LAEs,
and the fild arotund the Great Observatory Optical Deep Survey North
(GOODS-N) with $0.24$ deg$^2$ and $185$ LAEs. 

There is abundant observational work done on LAEs at redshift $z=3.1$
\citep{Kudritzki2000,Matsuda2005,Gawiser2007,Nilsson2007,Ouchi2008}.
However, we decide to focus on the data from \cite{Yamada2012} because
it has the largest covered area with homogeneous intrumentational
conditions (telescope, narrow band filter), data reduction pipeline
and conditions to construct the LAE catalog. This ensures that the
number density variations among fields are due only to
astrophysical reasons and not different observational conditions or
criteria to construct the catalogs.

\subsection{Simulation and Halo Catalogs}

The Bolshoi simulation \citep{Bolshoi} we use in this paper was
performed in a cubic volume of 250 $h^{-1}$ Mpc on a side. It includes
dark matter distribution is sampled using $2048^{3}$ particles, which
translates into a particle mass of $m_{\rm   p}=1.35\times 10^{8}$
$h^{-1}$ M$_{\odot}$.  The cosmological parameters are consistent with
a WMAP5 and WMAP7 data with a  density $\Omega_{\rm m} = 0.27$,
cosmological  $\Omega_{\Lambda}=0.73$, dimensionless Hubble constant
$h=0.70$, slope of the power spectrum $n=0.95$ and normalization of the
power spectrum$\sigma_{8}=0.82$ \citep{Komatsu2009,Jarosik2011}.  

We use halo catalogs constructed with a Friend-of-Friends (FOF)
algorithm with a linking lenght of 0.17 times the interparticle
distance. The minimum halo mass in the models we construct in this \documentname
correspond to groups of $\sim 75$ particles. The catalogs were
obtained from the publicly available Multidark database \footnote{{\tt
    http://www.multidark.org/MultiDark/}} \citep{2011arXiv1109.0003R}.
We focus our work on halos more massive than $1\times 10^{10}$\hMsun
that are resolved with at least $70$ particles.


\subsection{A Model to Populate Halos with LAEs}
\label{subsec:mocks}

In our model a dark matter halo can only host one or zero LAE. 
There are three parameters in the model that decide whether a halo can host a
galaxy or not: the lower and upper bounds for the mass range $M_{\rm min}<
M_{\rm h} < M_{\rm max}$ where LAEs reside and the fraction $f_{\rm
  occ}$ of such halos that effectively host a LAE. We do not assign a
luminosity to each LAE. We are primaraly interested in
constraining the halo mass range hosting detectable LAEs under the
conditions defined by \cite{Yamada2012}. In what follows will describe by
the letter ${\mathcal M}$ a model defined by an specific choice of the
three scalar parameters $M_{\rm min}$, $M_{\rm  max}$ y $f_{\rm occ}$.  


For each model ${\mathcal M}$ we create a set of mock fields from
disjoint volumes in the simulation. Each volume has the same geometry
probed by Suprime-CAM and the narrow band filter, namely rectangular
cuboids of dimensions $46\times 35\times 41$ $h^{-3}$Mpc$^{3}$ where
the last dinemsion goes in the redshift direction. This corresponds to
a total area of $880$ arcmin$^{2}$ in each mock field. We construct a total
$5\times 7 \times 6=210$ of such volumes from a snapshot in the Bolshoi
simulation. In each mock field a LAE is assigned to the position of a
dark matter halo if the halo mass is in the range allowed by the model
$M_{\rm min}<M_{\rm h}<M_{\rm max}$ and a random variable taken from
an homogeneous distribution $0\leq \xi<1$ is smaller than the occupation
fraction $\xi<f_{\rm occ}$.

Next we construct mock surveys by making groups of $11$ mock fields
out of the $210$ available volumes. In total $15$ mock surveys are
constructed for each model $\mathcal{M}$. The grouping is done in two
different ways. In the first way, called {\texttt match}, we follow
the clustering of the observed fields. From the $11$ mock fields, $7$
are constructed from contigous fields in the simulation to mimic the
SSA22 region, $3$ are also contiguous between them but not to the first
$7$ fields to mimic the SXDS fields and finally $2$ non-contigous
fields to imitate the SDF and GOODS-North field. Our main goal with
this selection is to test the impact on the final statistics of having
$7$ clustered fields. The second way to group the mock fields is called {\texttt
  random}, whereby all the $11$ fields are selected in such a way as
to avoid that any two volumes are contigous.


\subsection{Sampling and Selecting Models}

We make a thorough exploration of the parameter space for the models
${\mathcal M}$. $\log_{10} M_{\rm min}$ takes $30$ values from $10.0$ up
to $12.9$ with an even spacing of $0.1$ dex. $\log_{10} M_{\rm max}$
takes values in the same range as $\log_{10}M_{\rm min}$ only with a
displacement of $0.1$ dex in the whole range. The occupation fraction
$f_{\rm occ}$ takes $10$ different values from $0.1$ to $1$ regulary
spaced by $0.1$. In total the number of different models ${\mathcal
  M}$ that are explored is $30 \times 30 \times 10 = 9000$.

For each mock survey generated in a given model ${\mathcal M}$ we
compute the surface density in the $12$ mock fields. We perform a
Kolmogorov-Smirnov (KS) to compare this mock date against the $12$
observational values. From this test we obtain a value $0<P<1$ to
reject the null hypotesis, namely that two data sets come from the
same distribution. In this paper we consider that for values $P>0.05$
the two distributions can be thought as coming from the same
distribution.

In this paper we consider that a model
${\mathcal M}$ that has at least one (1) mock survey (out of 15)
consistent with the observed distribution of LAE number densities has
viable parameters thatdeserve to be considered for further analysis.  

\section{Basic Results}


\begin{figure}
\begin{center}
\includegraphics[width=1.10\linewidth,angle=0]{./plots/Fig1.pdf}
\caption{ \label{fig:halos} Surface density of dark
  matter halos as a function of a minimum halo mass to count the
  total number of elements in a volume. Each line represents on of the
  $210$ volumes of dimensions $46\times 35\times 41$ h${-3}$Mpc$^{3}$
  in the Bolshoi simulation. The horizontal grey band represents the
  range of surface densities observed for LAEs at $z=3.1$ as reported
  by \citep{Yamada2012}.}
\end{center} 
\end{figure}


\subsection{Dark Matter Halo Number Density}

In Figure \ref{fig:halos} we present the results for  the
integrated dark matter halo surface density as a function of halo
mass. Each line corresponds to one of the 210 sub-volumes in the
Bolshoi simulation. The gray band indicates the surface density
values for LAEs allowed reported in observations \citep{Yamada2012}.
 
This result provides the basis to understand why a range of models
${\mathcal M}$ can be expected to be consistent with
observations. In Figure \ref{fig:halos} we can read that models
with a minimum mass $\log_{10} M_{\rm min}>11.5$\hMsun will always have a
surface number density lower than the observational
constrain. The oppossite is true in models with $\log_{10} M_{\rm min}<10.5$
that will show surface number density larger than observations, this
implies that in such models the occupation fraction has to be tuned
$f_{\rm occ}<1.0$ as to lower the halo numeber density to match the
gray band values.

Conversely, there are regions in the plot where the halo surface
density is always higher than the observational constraints correspond
to models ${\mathcal M}$ with a minimum mass below $M_{\rm min}<
3\times 10^{10}$\hMsun. Models with this minimum mass hava a chance
for successfuly reproducing observations if the occupation fraction
$f_{\rm occ}<1$ is tuned as to lower the halo number density down to
the observed value.    

In the next subsection we quantify this intuition by means of the KS
tests between mock surveys and observations.

\subsection{Three Regions in Parameter Space}

\begin{figure*}
\begin{center}
\includegraphics[width=0.46\linewidth,angle=0]{./plots/Fig2_match_P5.pdf}
\vspace{5mm}
\includegraphics[width=0.49\linewidth,angle=0]{./plots/Fig3_match_P5.pdf}\\
\includegraphics[width=0.46\linewidth,angle=0]{./plots/Fig2_random_P5.pdf}
\hspace{5mm}
\includegraphics[width=0.49\linewidth,angle=0]{./plots/Fig3_random_P5.pdf}\\
\end{center} 
\caption{$M_{\rm min}$-$M_{\rm max}$ (left) and $M_{\rm
    min}-f_{\rm occ}$ (right) planes for all models with
  $P>0.05$ in two different ways used to construct the mock
  surveys. The color code corresponds to the number of mock surveys
  that are found to be compatible with observations in terms of the KS
  test with $P>0.05$. Only regions of parameter space with at least
  one (1) consistent mock survey are included. \label{fig:landscape}}   
\end{figure*}


Figure \ref{fig:landscape} presents regions in parameter space $M_{\rm
min}-M_{\rm max}$, $M_{\rm min}-f_{\rm occ}$ where the KS test yields
values of $P>0.05$ at least for one mock survey. For those models it
is not possible reject the hypothesis that the simulated and observed
data for the surface number density come from the same parent
distribution.

The upper (lower) panels correspond to the {\texttt match} ({\texttt
  random}) method to build the mock surveys from individual
fields. The plot shows number of mock surveys consistent
with observations. There are between $550$ to $600$ models out of the
original $9000$ models that have at least one (1) mock survey
consistent with observations. 


In Figure \ref{fig:landscale} there are three regions of parameter
space that can be clearly distinguised. The first region corresponds
to models where the minimum mass is high $\log_{10}M_{\rm min}>
11.5$. None of this models is compatible with observations as expected
from the results in the previous section. For these models the number
density of LAEs is too low. 

The second region corresponds to an intermediate range for the minimum
mass $10.5<\log_{10}M_{\rm min}<11.5$ where regardless of the value of
the maximum mass $M_{\rm max}$ it is possible to tune the occupation
fraction $f_{\rm occ}$ to bring some of the mock observations into
good agreement with observations. In this region in parameter space
one can thus find two extreme kinds of models.  One kind where the
mass interval is very narrow with sizes smaller than $<0.3$ dex (a
factor of two in mass) and others where the mass interval is very
extended, larger than $1.0$ dex, going up to the maximum halo mass
present in the simulation at that redshift. 


The third region in parameter space corresponds to $\log_{10}M_{\rm
  min}<10.5$. In this case only models with a narrow mass interval of
at most $0.5$ dex ($\log_{10}M_{\rm max}<11.0$) and low
occupation fractions $f_{\rm occ}<0.3$ are allowed. 

Without any additional information our method allows us to infer that
most of the succesful models are found in the second and third region of
parameter space where. This result was already expected from halo
abundance calculations shown in Figure \ref{fig:halos}. In the next
section we reduce the size of this region by using more stringent
constraints to define the agreement with observations and including
additional observational information onpossible values for the
occupation fraction at $z\sim 3$. 

\section{Additional Constraints}

\begin{figure*}
\begin{center}
\includegraphics[width=0.46\linewidth,angle=0]{./plots/Fig4_match_P5.pdf}
\hspace{5mm}
\includegraphics[width=0.46\linewidth,angle=0]{./plots/Fig4_random_P5.pdf}
\end{center} 
\caption{ Number of models with a mininium number of mock survey
  realizations that are consistent with observations.
  \label{fig:high_success_rate}.}  
\end{figure*}
 
In this section we consider three additional constraints on the models
we accept as succesul. First by taking models with the
highest possible number of mock surveys consistent with
observations. Second by including additional observational constraints
on the occupation fraction for high redshift LAEs. Third by comparing
the angular correlation function against observational constraints by
\cite{Hayashino2004}. 

\subsection{Models with the highest success rates}

For each model ${\mathcal M}$there are 15 different mock surveys. In the
previous section we presented the models that had at least one (1)
mock survey with $P>0.05$.

Figure \ref{fig:high_success_rate} shows the number of models
that have at least $N_{\rm high-P}$ mocks with $P>0.05$ for the {\texttt
  match} and {\texttt random} methods.  This shows that there are
here are between $80$ to $100$ models with all the 15 realizations with
$P>0.05$. This a reduction of a factor of $\sim 6$ with repect to the
total number of mocks with at least one consistent mock. 

Figure \ref{fig:restriction_mock} presents the locii of these models
in the parameter space $M_{\rm min}-M_{\rm max}$ and $M_{\rm
  min}-f_{\rm occ}$ for the {\texttt match} method, the results for
the {\texttt random} method are similar. In this case the models with
$\log_{10}M_{\rm min}< 11.7$ are greatly reduced. This corresponds to
the regions in the parameter space in Figure \ref{fig:landscape} that
already had a low number of consistent mock surveys. From the right
panel in Figure \ref{fig:restriction_mock} one can see that there is
not a strong selection effect on the occupation fraction $f_{\rm
  occ}$. 

\subsection{Additional  Constraints on the Occupation Fraction}

We impose an additional restriction using the observational results
by \cite{Hayes2010}. They constrained the value of $f_{\rm
  occ}$ at $z=2.2$ to be $f_{\rm occ}=0.10$. This estimation was based
on blind surveys of the H$\alpha$ and Lyman $\alpha$ line with the European Southern
Observatorio (ESO) Very Large Telescope (VLT). Using corrections by
extinction to obtain an estimate for the intrinsic H$\alpha$
luminosity, and using values for the theoretical expectation of the
ratio Lyman$\alpha$/H$\alpha$ they derive an bulk escape fraction for
the Lyman$\alpha$ radiation of $f_{\rm esc}=(5.3\pm 3.8)\%$ or $f_{\rm
esc}=(10.7\pm 2.8)\%$ if a different dust correction is used. The
authors show that the luminosity function for LAEs at $z=2.2$ is
consistent with the escape fraction being constant for every galaxy
regardless of its luminosity. From this results they derive that
almost $90\%$ of the star forming galaxies emit insufficient
Lyman $\alpha$ to be detected, effectively setting the occupation
fraction to be $f_{\rm occ}=0.10$. For the cosmological parameters
used in this \documentname the age of the universe between $z=3.1$ and
$z=2.2$ has changed by $\sim 1$ Gyr. We assume that the physical
conditions that determine the escape fraction $f_{\rm esc}$ and the
occupation fraction $f_{\rm  occ}$ remain similar over that time
scale.

We limit the occupation fraction to
be in the range $f_{\rm  occ}\leq 0.2$ to allow for some flexibility on the time evolution
and the uncertainty in the dust correction used to infer the escape fraction. 
Figure  \ref{fig:restriction_mock_and_f_occ}  shows the prefered
models in the planes $M_{\rm min}-M_{\rm  max}$ and $M_{\rm
  min}-f_{\rm occ}$ for the {\tt match} and {\tt   random} methods.
With this additional constraint between $35$ to $40$ models are
consistent with observations, this represents a factor of $\sim 2$
reduction. All the models with $\log_{10}M_{\rm min}\geq 11.0$ are now
excluded and the best models are now clustered around a narrow region
in parameter space. %The list for the model parameters is in the
%Appendix in Table \ref{table:models_match}. 

\begin{figure*}
\begin{center}
\includegraphics[width=0.46\linewidth,angle=0]{./plots/Fig5_match_mass_mock.pdf}
\hspace{5mm}
\includegraphics[width=0.46\linewidth,angle=0]{./plots/Fig5_match_f_occ_mock.pdf}
\end{center} 
\caption{Favored regions in parameter space when the constraints on
  the maximal number of consistent mocks is imposed. The results for
  the {\texttt random} methodology (not shown here) are very similar to the ones
  presented here for the {\texttt match} method.
  \label{fig:restriction_mock}}  
\end{figure*}


\begin{figure*}
\begin{center}
\includegraphics[width=0.46\linewidth,angle=0]{./plots/Fig5_match_mass_mock_and_f_occ.pdf}
\hspace{5mm}
\includegraphics[width=0.46\linewidth,angle=0]{./plots/Fig5_random_mass_mock_and_f_occ.pdf}
\end{center} 
\caption{Best models when both the constraints on the maximal number
  of  consistent mocks and the occupation  fraction $f_{\rm occ}\leq
  0.2$ are included.  \label{fig:restriction_mock_and_f_occ}}   
\end{figure*}



\subsection{Consitency with the Angular Correlation Function}

The measurementens presented in \citep{Yamada2012} do not present a
new measurement of the angular correlation function (ACF). However
\citep{Hayashino2004} performed such calculation on the LAEs observed
in the densest field of SSA22.  

we calculate the angular correlation function
(ACF) for all the models having the $15$ mock surveys consistent with
observations. We perform this calculation as a sanity check and
potential additional constraint. The ACF is computed only the densest
subfield in all the 15 mock surveys corresponding to the SSA22
region. 


It is important to keep in mind that there are some differences
between this work and \citep{Yamada2012}. The color selection by
\cite{Yamada2012} is less stringent compared to the one by
\cite{Hayashino2004}. Also the EW threshold is different,
\cite{Hayashino2004} uses a cut around $154$\AA instead of $190$\AA
used by \citep{Yamada2012}

The comparison between the simulated and observed ACFs is also done using a
$\chi^2$ statistic which includes the information on the measurement
uncertainties and the statistical dispersio from the $15$ mock surveys
in the simulation

\begin{equation}
\chi^{2} = \sum_{\theta_i} \frac{(\xi_{\rm obs}(\theta_i) - \xi_{\rm
    sim}(\theta_i))^2}{\sigma_{\rm obs}^2(\theta_i) + \sigma_{\rm
    sim}^2({\theta_i})}, 
\end{equation}

where the sum is done over all the $13$ angle values $\theta_i$ where the
ACF has been computed. In Figure XX we plot this statistic for all the
models considered. In all cases $\chi^{2}<7$ which is not sufficient
to consider that the correlation functions from the simulations are
statistically different from the observational one. The constribution
of cosmic variance in the simulations together with the statistical
uncertainties make all tested models compatible with the observational
constraints. 

%From these tests we conclude that the ACF on small fields does not
%provide additional constraints to further select models for halos
%hosting LAEs. The reason is that 






%Given that the ACF reported by Hayashino et al en 2004 is taken over
%the densest  field oserved in the SSA22 region by  Yamada et el in
%2012 it is expected that the predicted ACF in the SSA22 region should
%reproduce this observation. In the  left panel of figure
%\ref{figure:correlation_match} we can see the predicted ACFs  an their
%corresponding standard deviation over the seven fields that mock the
%SSA22  region. It can be seen that the model with $M_min=10.6$ seems
%to better reproduce the  Hayashino's ACF and that the corresponding
%field is in fact an overdense field in the SSA22 region covered by
%Yamada et al. 

We also present the results for the mean angular correlation function in
therms of the angular correlation lenght obtained  by fitting to a power-law
function:  

\begin{equation}
\xi(\theta) = \left(\frac{r}{\theta_{0}}\right)^{-\gamma}
\end{equation}

The fitting is done to both the mean ACF and the observational
ACF reported by \cite{Hayashino2004} .  The results are shown 
in Figure  \ref{figure:correlation_parameters}  in a $\theta_{0}-\gamma$ 
plane where the average and standard deviation for each mock 
are shown in comparison with the result derived from observations. 
Error bars in these figures represents the standard deviation of the
ACF over all the sub-fields.  

It can be seen in the left panel figure \ref{figure:correlation_parameters} that
The observational ACF measured by \cite{Hayashino2004} (green dot) set aditional 
constrains on the parameters models. Only models with angular-correlation lenght within  
$\theta_{0}\sim[15,23]arcsec$ reproduces observations and this rule out several models. 

In order to identify the models that are consistent with observation we
plot if the right panel of figure \ref{figure:correlation_parameters} 
$\theta_{0}$ vs $M_{\rm min}$ and divide the plots into two sets: Those models
with $\log \frac{M_{\rm max}}{M_{\odot}}<12$ (blue dots) and those with 
$\log \frac{M_{\rm max}}{M_{\odot}}>12$ (green dots). The red rectangle includes
the parameter region wich is consistent with the observational constraint in $\theta_{0}$.
With this restriction we find that models with $\log \frac{M_{\rm max}}{M_{\odot}}>12$ and
$\log \frac{M_{\rm min}}{M_{\odot}}>11.1$can be ruled out.

%In general, we observe that the standard
%deviation of the computed ACF in the subfields increases with
%$M_{min}$ following the same trend as in Figure
%\ref{figure:laes_dist}, as a direct consequence of cosmic variance.  

In figure \ref{fig:restriction_mock_and_f_occ_corr} we present the prefered
models in the planes $M_{\rm min}-M_{\rm  max}$ and $M_{\rmmin}-f_{\rm occ}$
for the {\tt match} method after applying the observational constraints 
in the occupation fraction and the ACF.



Finally, we also present the ACF in the case of the full SSA22
region which has been homogeneously observed by \citep{Yamada2012}. To
this date the observational ACF has not been reported in the
litereature, therefore our calculations can be considered as
predictions. 

In Figure X we present the results for the models. The full list of
these correlation functions can be found in the the data repository
for this paper in \verb"github". 

\begin{figure*}
\begin{center}
\includegraphics[width=0.46\linewidth,angle=0]{./plots/power_law_correlation.pdf}
\hspace{5mm}
\includegraphics[width=0.46\linewidth,angle=0]{./plots/mmin_vs_correlation.pdf}
\end{center}
\caption{Left: The fitted parameters  ($\theta_{0}$ vs $\gamma$) of the 
mean ACF  assuming a powerlaw behavior. Each blue dot corresponds to a
match-model while the green dot represent the powerlaw fit of the observational
ACF reported by \citet{Hayashino2004}.  The error bars represents the dispersion
in each parameter over all the mock surveys for a given model and dispersion due
to the observational error for the measured ACF by \citet{Hayashino2004}. Rigth: 
$\theta_{0}$ vs $M_{min}$. The date is subdivided into two blocks $M_{max}<12.0$ 
(blue) and $M_{max}>12.0$ (green). The red rectangle includes the region within
models are consisted with the observational angular-correlation lenght
($15<\theta_{0}<23$) as it can be deduced from the left panel. \label{figure:correlation_parameters}} 
\end{figure*} 


\begin{figure*}
\begin{center}
\includegraphics[width=0.46\linewidth,angle=0]{./plots/mmin_vs_mmax.pdf}
\end{center}
\caption{Best models when both the constraints on the maximal number
  of  consistent mocks and the occupation  fraction $f_{\rm occ}\leq
  0.2$ are included. \label{fig:restriction_mock_and_f_occ_corr}} 
\end{figure*} 
%\begin{figure*}
%\begin{center}
%\includegraphics[width=0.46\linewidth,angle=0]{./plots/random_large_correlation_selected_models.pdf}
%\hspace{5mm}
%\includegraphics[width=0.46\linewidth,angle=0]{./plots/random_full_correlation_selected_models.pdf}
%\end{center} 
%\caption{ mean ACFs   and their correponding standard deviation (error
%  bars)  of some selected models in different mass ranges over the 7
%  subfields of the SSA22 field (left) and the entire 12 field sample
%  (rigth) using the  random
%  configuration. \label{figure:correlation_random} }   
%\end{figure*}

\section{Discussion}



%\begin{figure*}
%\begin{center}
%\includegraphics[width=0.49\linewidth,angle=0]{./plots/mytest.pdf}
%\includegraphics[width=0.49\linewidth,angle=0]{./plots/mytest.pdf}
%\end{center} 
%\caption{Spatial distribution for two mocks corresponding to the model
%  $M_{\rm min}=$, $M_{\rm max}=$ and $f_{\rm occ}=$. All the 15
%  different mock  surveys for this model in the {\texttt match}
%  configuration are consistent with observations at the $P>0.05$
%  level. The full data for all the mocks can be found in the github
%  repository for this paper.
%  \label{figure:spatial_distro}.}
%\end{figure*}


When we include the tightest constraints on the mock catalogs, we find
that there are 30 set of parameters of our model, out of the original
90000 initial models, that are consistent with the observational
constraints at redshift $3.1$: the distribution of the number density,
the inferred values for the average occupation fraction. The
consistency with the angular correlation, in terms of the $\chi^2$
statistics did not help to discard any additional models with a
significant degree of confidence. 


These 30 models can be classified into two families of the same
size. The first, where the range $M_{\rm min}-M_{\rm max}$ is narrow,
typically of less than $<1.0$ dex. While in the second familiy the
extent $>1.0$ dex. In the first case the minimum halo mass is found to
be in a wide range $10^{10}\hMsun <M_{\rm min}< 10^{11.5}\hMsun$ while
in the second case, only models with $M_{\rm min}\sim 10^{10.9}\hMsun$
are compatible with the observational contraints. In what follows we
discuss the implications of the existence of these two families of
models.  

When we additional include the ACF constraint we find that only models with 
$\log_{10}M_{\rm min}\leq 11.0$ and $\log_{10}M_{\rm max}\leq 12.0$ 
are now consistent with observations.

Including all the observational constraints we find that only between $15$ to $20$ 
models reproduces observations. The list for the model parameters is in the
Appendix in Table \ref{table:models_match}.

\subsection{Implications for galaxy formation models}

In the case of a narrow of masses to host LAEs the upper masses are
bound to be $M_{\rm max}< 10^{11.5}$\hMsun as it is show in Figure
\ref{figure:mock_and_f_occ}. For halos more massive than this bound it
is naturally expected that the galaxies can be observed as Lyman Break
Galaxies (LBGs). This would imply that not all the bright LBGs can
detected as a LAE.

We have the opposite situation in the second familiy of models. If
we have a wide range in halo masses, where the upper end of the halo
masses can be considered as observed LAEs, one can expect that bright
LBGs will have a correspondence as observed LAEs. The most interesting
aspect is that there is a clear cut in the minimal mass that can be
attained by observed LAEs $M_{\rm min}> 10^{11}$\hMsun. This puts a tight
constraint on the relationship between the minimum star formation rate
required to be observed as a LAE and this minimal halo mass.


... Intrinsic emission and escape fraction.

... Star formation rate efficiency at this redshift.

... Mass dependence of the escape fraction.

... Discuss all this in terms of the star formation efficiency in
Behroozi et al.

\subsection{Implications for large LAEs surveys}

... The bias for the preferred halo mass.

... The scale at which cosmic variance drops.

... This can be observationally tested with HETDEX.

\subsection{On the reproducibility of our results}

... All the software to produce the results in this paper is publicly
available. 

... The raw catalogs can be obtained from the MultiDark database but
can also be obtained in the repository of this paper on github.

\section{Conclusions}
In this \documentname we constrain the preferred mass for dark matter
halos hosting Lyman Alpha Emitters at a redshift $z=3.1$. We use a
method that matches the cosmic variance in the surface
density number of LAEs between mock and real observations. The mock
catalogs are based on a simplified model with three basic parameters: the halo
mass range where LAEs can be found, $M_{\rm   min}<M_{\rm h}<M_{\rm
  max}$, and the fraction of the halos in thisrange that are actully
occupied, $f_{\rm occ}$. After a thorugh exploration of the parameter
space we are able to constrain the mass range of dark matter halos
hosting LAEs to be in the range $<M_{\rm   h}<$ and a corresponding
occupation fraction that escales as $f_{\rm   occ} = M_{\rm min}$. 

We use three additional constraints to reduce the allowed
range of models. The first imposes a tighter criterion to consider a
model succesful, namely that all the mock surveys for a given model
must be consistent with observations. This restriction narrows down
the allowed range of models to be. 

The second constraint is based on the observational results that high
redshift LAEs have a bulk Lyman alpha escape fraction of $XX$ which
can be also interepreted as an average occupation fraction of $XX$.

Including additional observational constraints on the occupation
fraction allows us to reduce the range of allowed halo masses to be in
a narrower range of $<M_{\rm h}<$. Including the information from the
angular correlation function (ACF) does not allows us to impose
further constraints. This is due to the scatter in the ACF due to the
cosmic variance on the field observed by XXX


We simulation allows us to extract $210$ sub-boxes each of which has a
comparable volume to the individual fields of view observed by
\cite{Yamada2012}. The comparison of the observed number density
distribution against the results from our model is based on three
different ways of constructing mock surveys. The first reproduces the
spatial correlation between the $12$ observational fields ({\texttt
  match}), the second breaks this spatial correlation while keeping
the number of fields ({\texttt random}) and the third one simply
includes all the $210$ sub-boxes ({\texttt full}). We find that the
methods {\texttt match} and {\texttt random} allow a larger set of
models than the {{\texttt random}} method. We do not find a
significant difference between the two first methods. 


\section*{Acknowledgments} 


\begin{table}
\begin{center}
\begin{tabular}{ccc}\hline\hline
$\log_{10}M_{\rm min}$ & $\log_{10}M_{\rm max}$ & $f_{\rm occ}$\\\hline
10.1& 10.2& 0.1\\
10.3& 10.5& 0.1\\
10.4& 10.7& 0.1\\
10.5& 10.9& 0.1\\
10.5& 11.0& 0.1\\
10.6& 11.2& 0.1\\
10.6& 11.3& 0.1\\
10.6& 11.4& 0.1\\
10.6& 11.5& 0.1\\
10.6& 11.6& 0.1\\
10.6& 11.7& 0.1\\
10.6& 10.8& 0.2\\
10.7& 11.0& 0.2\\
10.8& 11.2& 0.2\\
10.8& 11.3& 0.2\\
10.8& 11.4& 0.2\\
10.9& 11.7& 0.2\\
10.9& 11.8& 0.2\\
10.9& 11.9& 0.2\\
%10.9& 12.0& 0.2\\
%10.9& 12.1& 0.2\\
%10.9& 12.2& 0.2\\
%10.9& 12.3& 0.2\\
%10.9& 12.4& 0.2\\
%10.9& 12.5& 0.2\\
%10.9& 12.6& 0.2\\
%10.9& 12.7& 0.2\\
%10.9& 12.8& 0.2\\
%10.9& 12.9& 0.2\\
%10.9& 13.0& 0.2\\\hline\hline
\end{tabular}
\end{center}
\caption{List of model parameters for the best models that have all
  mock surveys consistent with observations and an occupation
  fraction $f_{\rm occ}\leq 0.2$. This corresponds to the {\texttt
    match} method to construct the mock surveys. \label{table:models_match}. }
\end{table}

\bibliographystyle{mn2e}
\bibliography{references} 

\end{document}

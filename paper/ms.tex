\documentclass[usenatbib]{mn2e}
\usepackage{amsmath} 
\usepackage{amssymb} 
\usepackage{graphics}
\usepackage{graphicx}
\usepackage{epsfig}  
\def\be{\begin{equation}}
\def\ee{\end{equation}}
\def\ba{\begin{eqnarray}}
\def\ea{\end{eqnarray}}

\newcommand{\documentname}{paper~}
\newcommand{\match}{{\tt match}~}
\newcommand{\apj}{ApJ}  
\newcommand{\apjs}{ApJS}  
\newcommand{\apjl}{ApJL}  
\newcommand{\aj}{AJ}  
\newcommand{\mnras}{MNRAS}  
\newcommand{\mnrassub}{MNRAS accepted}  
\newcommand{\aap}{A\&A}  
\newcommand{\aaps}{A\&AS}  
\newcommand{\araa}{ARA\&A}  
\newcommand{\nat}{Nature}  
\newcommand{\physrep}{PhR}
\newcommand{\pasp}{PASP}    
\newcommand{\pasj}{PASJ}    

\newcommand{\kms}{\,km~s$^{-1}$}
\def\squig{\sim\!\!}
\newcommand{\LCDM}{$\Lambda$CDM~}
\newcommand{\beq}{\begin{eqnarray}}  
\newcommand{\eeq}{\end{eqnarray}}   
\newcommand{\zz}{$z\sim 3$} 
\newcommand{\avg}[1]{\langle{#1}\rangle}  
\newcommand{\ly}{{\ifmmode{{\rm Ly}\alpha}\else{Ly$\alpha$~}\fi}}
\newcommand{\hMpc}{{\ifmmode{h^{-1}{\rm Mpc}}\else{$h^{-1}$Mpc }\fi}}  
\newcommand{\hGpc}{{\ifmmode{h^{-1}{\rm Gpc}}\else{$h^{-1}$Gpc }\fi}}  
\newcommand{\hmpc}{{\ifmmode{h^{-1}{\rm Mpc}}\else{$h^{-1}$Mpc }\fi}}  
\newcommand{\hkpc}{{\ifmmode{h^{-1}{\rm kpc}}\else{$h^{-1}$kpc }\fi}}  
\newcommand{\hMsun}{{\ifmmode{h^{-1}{\rm
        {M_{\odot}}}}\else{$h^{-1}{\rm{M_{\odot}}}$}\fi}}   
\newcommand{\hmsun}{{\ifmmode{h^{-1}{\rm
        {M_{\odot}}}}\else{$h^{-1}{\rm{M_{\odot}}}$}\fi}}   
\newcommand{\Msun}{{\ifmmode{{\rm {M_{\odot}}}}\else{${\rm{M_{\odot}}}$}\fi}}  
\newcommand{\msun}{{\ifmmode{{\rm {M_{\odot}}}}\else{${\rm{M_{\odot}}}$}\fi}}  
\newcommand{\clara}{{\texttt{CLARA}}~}
\newcommand{\rand}{{\ifmmode{{\mathcal{R}}}\else{${\mathcal{R}}$ }\fi}}  
\newcommand{\Lsun}{\mbox{\,$L_{\odot}$}}
\newcommand{\like}{\mathscr{L}}
\newcommand{\bftheta}{\mathbf{\Theta}}
\newcommand{\degree}{\ensuremath{^\circ}}
\def\spose#1{\hbox to 0pt{#1\hss}}
\def\simlt{\mathrel{\spose{\lower 3pt\hbox{$\mathchar"218$}}
     \raise 2.0pt\hbox{$\mathchar"13C$}}}
\def\simgt{\mathrel{\spose{\lower 3pt\hbox{$\mathchar"218$}}
     \raise 2.0pt\hbox{$\mathchar"13E$}}}
\font\smcap=cmcsc10

\begin{document}

\title[Halo mass and occupation fraction for LAEs at
  $z=3.1$]{Mass and occupation
  fraction of dark matter halos hosting Lyman-$\alpha$ emitters
  at $z\sim 3$}    
\author[~J.~E. Forero-Romero and ~J.~E Mejia-Restreo]{
\parbox[t]{\textwidth}{\raggedright 
Jaime E. Forero-Romero$^{1}$ and
Julian E. Mej\'ia-Restrepo$^{2}$ 
}
\vspace*{6pt}\\
$^{1}$ Departamento de F\'{i}sica, Universidad de los Andes, Cra. 1
No. 18A-10, Edificio Ip, Bogot\'a, Colombia \\
$^{2}$ Departamento de Astronom\'{i}a, Universidad de Chile, Camino el
Observatorio 1515, Santiago, Chile} 

\maketitle

\begin{abstract}
%
We derive constraints on the mass and occupation fraction of dark
matter halos hosting \ly Emitting galaxies (LAEs) at a redshift of
$z=3.1$ by matching the number density and the angular
correlation function between mock and observed fields. We explicitly
take into account the cosmic variance on the typical observed field size by
constructing mock fields from a large cosmological N-body
simulation matching the observational geometries. To populate the
halos in the simulation we use a model where a dark matter halo with
mass in the range $M_{\rm   min}<M_{\rm h}<M_{\rm max}$ can only host one LAE with a
probability $f_{\rm occ}$. We find that the large scatter in the mocks, induced by
cosmic variance, makes the number density and spatial clustering
information insufficient to derive a unique and narrow set of values
for the parameters in the model. In particular, it is always possible
to find a model that is consistent with observations and has any
desired occupation fraction in the range $0.1\leq f_{\rm occ}\leq 1.0$
.  Nevertheless, in the models we can define three different families
of parameters based on their mass range, $\Delta M \equiv
\log_{10}M_{\rm max} - \log_{10} M_{\rm min}$, and occupation fraction,
$f_{\rm occ}$. The dominant family is composed by models with a narrow
mass range $\Delta M<1.0$ dex, a low occupation fraction $f_{\rm
  occ}\leq 0.3$ and a maximum mass $M_{\rm  max}< 10^{12}$\hMsun.  The
existence of this dominant family gives support to the idea that the
most massive dark matter halos at that epoch do not host the brightest
LAEs and that only a small fraction of star forming galaxies can be
actually detected as LAEs.
\end{abstract}

\begin{keywords}
{cosmology: theory – cosmology: large-scale structure of universe –
  galaxies: formation – galaxies: high-redshift – galaxies: statistics
  – galaxy: haloes} 
\end{keywords}


\section{Introduction}

Lyman-$\alpha$ emitting galaxies (LAEs) have become in the last decade
a  central topic in studies of structure formation in the Universe. They 
are helpful in a diverse range of fields. LAEs can be used as probes
of reionization \citep{Dijkstra11}, tracers of large scale structure
\citep{Koehler2007},  signposts for low metallicity stellar
populations, markers of the galaxy formation process through
cosmic history \citep{ForeroRomero2012} and tracers of active star
formation  \citep{Guaita2013}

At the same time, theoretical and observational developments have
contributed to the emergence of a paradigm to describe structure
formation in a cosmological context. In this context it is considered
that dominant matter content of the Universe is to be found in dark
matter (DM), whereby each galaxy is hosted by larger dark matter structure
known as a halo. \citep{Peebles1980,SpringelNature05}.

Most models of galaxy formation find that halo mass can be
used to predict galaxy properties such as the stellar mass and
star formation rate \citep{Behroozi2012}. Processes that regulate the
star formation cycle are also though to be strongly dependent on its
mass.  For these reasons, finding the typical dark matter halo mass
hosting LAEs represents a significant step forward to understand the
nature of this galaxy population in the context of Lambda Cold Dark
Matter ($\Lambda$CDM) paradigm.    

Some theoretical approaches to this problem have been based on
ab-initio approach. Starting from the DM halo population, the
corresponding intrinsic star formation properties are inferred
together with other statistics such as the luminosity function, the
correlation function and the equivalent width distributions. Such
modelling has been implemented from analytic considerations,
semi-analytic models  and  full N-body hidrodynamical simulations
\citep{Laursen2007, Dayal2009, ForeroRomero2011, Yajima2012, Orsi2012, Soler2012}. 

In addition to the uncertainties in the astrophysical processes
describing star formation in galactic populations, is the calculation
of the fraction of \ly photons that escape the galaxy to the observer
is another debated step. Given the resonant nature of the \ly line,
the radiative transfer of \ly is sensitive to the density,
temperature, topology and kinematics of the neutral Hydrogen in the
interstellar medium
(ISM). \citep{Neufeld1991,ForeroRomero2011,Dijkstra2012,Laursen2013,Orsi2012}.   

This problem can be tackeld with Monte Carlo simulations for the
radiative transfer, however the complexity of the relevant physical
processes makes it difficult to achieve a robust consensus on what is
the theoretical expected value for the \ly escape fraction
at high redshift. 

Another approach to infer the typical DM halo mass of halos hosting
LAEs is based on the spatial information. This uses the fact that in cold
dark matter cosmologies the spatial clustering of galaxies on large
scales is entirely dictated by the halo distribution
\citep{Colberg00}, which in turn has a strong dependence on halo
mass. Using measurements of the angular correlation function of LAEs,
observers have put constraints on the typical mass and occupation
fraction of the putative halos hosting these galaxies
\citep{Hayashino2004,Gawiser07,Nilsson2007,Ouchi2010}. In this
studies the observations were done on single fields of $\sim 1$
deg$^{2}$ and the conclusions derived on the halo host mass from
clustering information do not delve to deeply into the impact of
cosmic variance.

Recently \cite{Yamada2012} observed a wide area a bit more than ten
times the sky area covered so far by individual campaigns. The
observations were done with the same instrument and same criteria to
reduce the observational data and produce LAE catalogs. This large
data-set allows us to make the first study on the expected cosmic
variance effects in LAEs at $z=3.1$ and its impact on the
determination of the halo mass and occupation fraction. 


In this \documentname to study the impact of cosmic variance,
we implement a method to populate the DM halos in cosmological
simulations with LAEs. This approach bypasses all the uncertainty
involved in the estimation of star formation rates and \ly escape
fraction. The model only considers whether a DM can host a
detectable LAE or not. It does not predict a \ly  luminosity. Once the
mock catalogs are constructed  we proceed with a direct comparison
against the statistics derived from observations. This allows us 
to find the preferred mass range of DM halos hosting these galaxies
and its occupation fraction.




This \documentname is structured as follows. In the next section we present
the simulation and model used to produce the mock catalogs and the criteria
we use to compare them against observations. In \SS \ref{sec:results} we
present the main results for the halo mass and occupation fraction. We
continue with a discussion of these results under the light of other
observational and theoretical results. Finally, we present our
conclusions in \SS
\ref{sec:conclusions}. 

Throughout this \documentname we assume a $\Lambda$CDM cosmology with the
following values for the cosmological parameters, $\Omega_{m}=0.27$,
$\Omega_{\Lambda}=0.73$ and $h=0.70$, corresponding to the matter
density, vacuum density and the Hubble constant in units of 100 km
s$^{-1}$ Mpc$^{-1}$. 

\section{Methodology}

Our method is based on the comparison of observations and mock
catalogs. We use two different kinds of statistics to perform the
comparison: (i) the distribution of the surface number density
across fields fields and (ii) the angular correlation function measured in
some fields.

In the next subsections we describe in detail the four key
elements of this workflow. First, we present the observations we take
as a benchmark. Second, we describe the main characteristics of the
N-body simulation and the halo catalogs we use. Third, we recount the
important parameters of the simplified model that we use to populate
the halo catalogs with LAEs. Finally we describe some of the
statistical tests we adopt to compare observations and mocks.

\subsection{Observational Constraints}

The first observational constraint we use in this paper is the LAE number
density information at $z=3.1$ obtained by the panoramic narrow-band
survey presented by \cite{Yamada2012} from a survey
conducted with the Subaru 8.2m telescope and the Subaru Prime Focus Camera,
which has a field of view covering $34\times 27$ arcmin, corresponding to a
comoving scale of $46\times35$ Mpc $h^{-1}$ at $z=3.09$.  The narrow
band filter used in the survey is centered at $4977$ \AA with  $77$\AA width,
corresponding to the redshift range $z=3.062-3.125$ and $41$ Mpc
$h^{-1}$ comoving scale for the detection of the Lyman-$\alpha$ line
centered at $z=3.09$. The authors reported a
total  $2161$  LAEs with an observed equivalent width larger than $190$\AA 
over a total survey area of $2.42$ deg$^{2}$ that includes 12 subfields, 
this corresponds to average surface number density of $0.20\pm 0.01$
arcmin$^{-2}$.    

The survey covered four independent fields. The first is the SSA22
field of $1.38$ deg$^2$ with $1394$ detected LAEs (7 subfields), this field has been
known to harbor a region with a large density excess of galaxies. The
second observed region is composed by the fields Subaru/{\it
  XMM-Newton} Deep Survey (SXDS)-North, -Center and -South, with a 
total of $0.58$ deg$^2$ and $386$ LAEs (3 subfields). The third and fourth fields
are the Subaru Deep Field (SDF) with $0.22$ deg$^2$ and $196$ LAEs (1 subfield),
and the field around the Great Observatory Optical Deep Survey North
(GOODS-N) with $0.24$ deg$^2$ and $185$ LAEs (1 subfield). 

There is abundant observational work done on LAEs at redshift $z=3.1$
\citep{Kudritzki2000,Matsuda2005,Gawiser2007,Nilsson2007,Ouchi2008}.
However, we decide to focus on the data from \cite{Yamada2012} because
it has the largest covered area with homogeneous instrumentation
conditions (telescope, narrow band filter), data reduction pipeline
and conditions to construct the LAE catalog. This ensures that the
number density variations among fields are \emph{not} due to different
observational conditions or criteria to construct the catalogs.

The second constraint is the angular correlation function
(ACF). \cite{Yamada2012} does not report an ACF measurement. Instead
we use the results by \cite{Hayashino2004} and
\cite{Ouchi2008,Ouchi2010}. \cite{Hayashino2004} observed in the
densest field of SSA22 while \cite{Ouchi2008} observed on a 1 deg$^2$
sky of the SXDS Field.  There are some differences between these
observations and those by \cite{Yamada2012}. The details in the color
selection, corresponding limiting luminosities and EW thresholds are
different in these references. Nevertheless we use these observations
as additional constraints in spite of the fact that the first selected
models are based only on the surface density statistics by
\citep{Yamada2012}.  





\subsection{Simulation and Halo Catalogs}

The Bolshoi simulation \citep{Bolshoi} we use in this paper was
performed in a cubic volume of 250 $h^{-1}$ Mpc on a side. The
dark matter distribution is sampled using $2048^{3}$ particles, which
translates into a particle mass of $m_{\rm   p}=1.35\times 10^{8}$
$h^{-1}$ M$_{\odot}$.  The cosmological parameters are consistent with
a WMAP5 and WMAP7 data with a matter density $\Omega_{\rm m} = 0.27$,
cosmological constant $\Omega_{\Lambda}=0.73$, dimensionless Hubble constant
$h=0.70$, slope of the power spectrum $n=0.95$ and normalization of the
power spectrum$\sigma_{8}=0.82$ \citep{Komatsu2009,Jarosik2011}.  

We use halo catalogs constructed with a Friend-of-Friends (FOF)
algorithm with a linking length of 0.17 times the inter-particle
distance. The catalogs were obtained from the publicly available
Multidark database \footnote{{\tt
    http://www.multidark.org/MultiDark/}}
\citep{2011arXiv1109.0003R}. For each halo in the box we store its
position in the box (3-D coordinates) and FOF mass. We focus our work
on halos more massive than $1\times 10^{10}$\hMsun that are resolved
with at least $70$ particles, the reasons for this choice are
explained in the next sub-section.  


\subsection{A Model to Populate Halos with LAEs}
\label{subsec:mocks}

We assume that a dark matter halo can only host one
detectable LAE at most.  There are three parameters that
decide whether a halo host a LAE: the lower and upper bounds for the mass range $M_{\rm min}<
M_{\rm h} < M_{\rm max}$ where LAEs reside and the fraction $f_{\rm
  occ}$ of such halos that host a detectable LAE. The reader
must keep in mind that the physical interpreation of the occupation
fraction $f_{\rm occ}$ convolves two phenomena: the presence of a star
forming galaxy in a halo and its detectability as a LAE.  We do not
peform a explicit modeling for LAE detectability. Our model does not
assign a luminosity or escape fraction to each LAE.  We are only
interested in constraining the halo mass range hosting detectable LAEs
under the conditions defined by \cite{Yamada2012}.  


In what follows we note by the letter ${\mathcal M}$ a model
defined by a particular choice of the three scalar parameters $M_{\rm
  min}$, $M_{\rm  max}$ and $f_{\rm occ}$. For each model ${\mathcal
  M}$ we create a set of mock fields from disjoint volumes in the
simulation. Each volume has the same geometry probed by Suprime-CAM
and the narrow band filter, namely rectangular cuboids of dimensions
$46\times 35\times 41$ $h^{-3}$ Mpc$^{3}$ where the last dimension goes
in the redshift direction. This corresponds to a total area of $880$
arcmin$^{2}$ in each mock field. We construct a total $5\times 7
\times 6=210$ of such volumes from a snapshot in the Bolshoi
simulation. In each mock field a LAE is assigned to the position of a
dark matter halo if the halo mass is in the range allowed by the model
$M_{\rm min}<M_{\rm h}<M_{\rm max}$ and a random variable taken from
an homogeneous distribution $0\leq \xi<1$ is smaller than the occupation
fraction $\xi<f_{\rm occ}$.

Next we construct mock surveys by making groups of $12$ mock fields
out of the $210$ available volumes. In total $15$ mock surveys are
constructed for each model $\mathcal{M}$. The grouping of the $11$
mock fields into a mock catalog is done in two different ways. The
first is called {\texttt match}, it follows the clustering of the
observational fields. From the $12$ mock fields, $7$ are constructed
from contiguous fields in the simulation to mimic the SSA22 region,
$3$ are also contiguous between them but not to the first $7$ fields
to mimic the SXDS fields and finally $2$ non-contiguous fields to
imitate the SDF and GOODS-North field.  The second way to group the
mock fields is called {\texttt   random}, whereby all the $11$ fields
are selected in such a way as to avoid that any two volumes are
contiguous. In this \documentname we only report the results obtained
by the {\texttt match} method and mention explicitly differences
observed with the {\texttt random} selection. 





\subsection{Exploring and Selecting Good Models}

We make a thorough exploration of the parameter space for the models
${\mathcal M}$. $\log_{10} M_{\rm min}$ takes $30$ values from $10.0$ up
to $12.9$ with an even spacing of $0.1$ dex. $\log_{10} M_{\rm max}$
takes values in the same range as $\log_{10}M_{\rm min}$ only with a
displacement of $0.1$ dex in the whole range. The occupation fraction
$f_{\rm occ}$ takes $10$ different values from $0.1$ to $1$ regularly
spaced by $0.1$. In total the number of different models ${\mathcal
  M}$ that are explored is $30 \times 30 \times 10 = 9000$. 


The lower limit for the parameter $M_{\rm min}$ is set by the minium
occupation fraction we are able to consider. At $M_{\rm
  min}=10^{10}$\hMsun the halo number density is already $\sim 10$
times higher than the observational constraints for LAEs. This means
that models in that mass range and an occupation fraction $f_{\rm
  occ}=0.1$ have the possibility to be compatible with observations. Lower
values for $M_{\rm min}$ require $f_{\rm occ}<0.1$, which are not
considered in this \documentname. 

For each mock survey generated in a given model ${\mathcal M}$ we
compute the surface density in the $12$ mock fields. We perform a
Kolmogorov-Smirnov (KS) to compare this mock data against the $12$
observational values. From this test we obtain a value $0<P<1$ to
reject the null hypothesis, namely that two data sets come from the
same distribution. In this paper we consider that for values $P>0.05$
the two distributions can be thought as coming from the same
distribution.

We begin by considering that  a model ${\mathcal M}$ that has at least
one mock survey (out of 15) consistent with the observed
distribution of LAE number densities has viable parameters that
deserve to be considered for further analysis. Later on we impose
harder constraints to reduce the number of models by asking that all
the 15 mocks to be consistent with observations.

The second constraint  comes from the angular correlation function (ACF)
for all the models having the $15$ mock surveys consistent with
observations . The ACF is computed  ( using 
the Davis \& Peebles  estimator \citep{Davis1983} ) only on the densest
subfield in all the 15 mock surveys corresponding to the SSA22
region. We then take the mean ACF over the 15  mock surveys  for
each model as well as its the standard deviation to include
the effect of cosmic variance. 

We finally compare the observational ACF and the mean mock  ACF of the models   
in therms of the angular correlation lenght ($\theta_{0}$) by fitting 
them to a power-law function:  

\begin{equation}
\omega(\theta) = \left(\frac{\theta}{\theta_{0}}\right)^{-\beta}
\label{eq:fitting}
\end{equation}

Both mock and observational ACF are derived by a least
square minimization procedure. The observational errors of 
the ACF as well as the standard deviation of the mean modelled ACFs  are
considered to compute the error in the fitted parameters to ensure ourselves
to include poissionian and cosmic-variance errors. 
We consider that a model is consistent with observations if the two
parameters $\beta$ and $\theta_0$ are equal within a $1$-$\sigma$ range. 



 
\section{Results}
\label{sec:results}

The main purpose of this section is to show how different
observational constraints narrow down the parameters space of allowed
models. Each sub-sections presents the effect of adding new pieces of
observational or statistical evidence. 

\begin{figure}
\begin{center}
\includegraphics[width=1.10\linewidth,angle=0]{./plots/Fig1.pdf}
\caption{ \label{fig:halos} Surface density of dark
  matter halos as a function of a minimum halo mass to count the
  total number of elements in a volume. Each line represents one of the
  $210$ volumes of dimensions $46\times 35\times 41$ h${-3}$Mpc$^{3}$
  in the Bolshoi simulation. The horizontal gray band represents the
  range of surface densities observed for LAEs at $z=3.1$ as reported
  by \citep{Yamada2012}.}
\end{center} 
\end{figure}


\subsection{Dark Matter Halo Number Density}

In Figure \ref{fig:halos} we present the results for  the
integrated dark matter halo surface density as a function of halo
mass. Each line corresponds to one of the 210 sub-volumes in the
Bolshoi simulation. The gray band indicates the surface density
values for LAEs allowed reported in observations \citep{Yamada2012}.
 
This result provides the basis to understand why only a specific range of models
${\mathcal M}$ can be expected to be consistent with
observations. From Figure \ref{fig:halos} we can read that models with
a minimum mass $\log_{10} M_{\rm min}>11.5$\hMsun will always have a
surface number density lower than the observational
constrain. This makes impossible that models in that range can be
compatible with observations, there are simply to few halos compared
to observed LAEs.


The opposite is true in models with $\log_{10} M_{\rm min}<10.5$ that
have a surface number density larger than observations. In those cases
the occupation fraction $f_{\rm occ}<1.0$ can be tuned in order to
lower the halo number density to match observations.


Figure \ref{fig:halos} also illustrates the impact of cosmic
variance. At fixed mass there is an scatter of $0.3-0.6$ dex in the
mass range $10^{10}\hMsun < M_{\rm  min}<10^{11}\hMsun$.  A consequence of
this variation is that models with the same mass range and occupation
fraction can have mock fields with number densities varying up to a
factor of $\sim 2-5$. Also, the scatter in mocks across the simulation
is comparable to the $0.3$ dex scatter in observations.  


In the same Figure \ref{fig:halos} we see that at fixed number
density, around the range allowed by observations, there is an scatter
in the masses of $0.4-0.5$ dex. This intrinsic scatter in the masses
with a given number density adds is also included in the mock
construction process, enlarging the range of possible models
consistent with observations.  

\subsection{Models consistent with the surface density distributions}

\begin{figure*}
\begin{center}
\includegraphics[width=0.46\linewidth,angle=0]{./plots/Fig2_match_P5.pdf}
\vspace{5mm}
\includegraphics[width=0.49\linewidth,angle=0]{./plots/Fig3_match_P5.pdf}
%\includegraphics[width=0.46\linewidth,angle=0]{./plots/Fig2_random_P5.pdf}
%\hspace{5mm}
%\includegraphics[width=0.49\linewidth,angle=0]{./plots/Fig3_random_P5.pdf}\\
\end{center} 
\caption{$M_{\rm min}$-$M_{\rm max}$ (left) and $M_{\rm    min}-f_{\rm
    occ}$ (right) planes for all models with  KS test values
  $P>0.05$. The color code corresponds to the number of  mock surveys
  compatible with observations. Only regions of parameter  space with
  at least one consistent mock survey are  included. \label{fig:landscape}}     
\end{figure*}


Figure \ref{fig:landscape} presents regions in parameter space $M_{\rm
min}-M_{\rm max}$, $M_{\rm min}-f_{\rm occ}$ where the KS test yields
values of $P>0.05$ at least for one mock survey. For those models it
is not possible reject the hypothesis that the simulated and observed
data for the surface number density come from the same parent
distribution. In total, there are between $550$ to $600$ models out of
the original $9000$ models that have at least one mock survey
consistent with observations.  

In Figure \ref{fig:landscape} there are three regions of parameter
space that can be clearly distinguished. The first region corresponds
to models where the minimum mass is high $\log_{10}M_{\rm min}>
11.5$. None of these models is compatible with observations as expected
from the arguments presented in the previous subsection. For these
models the number density of LAEs is too low compared to observations.
 
The second region corresponds to an intermediate range for the minimum
mass $10.5<\log_{10}M_{\rm min}<11.5$ where, regardless of the value of
the maximum mass $M_{\rm max}$, it is possible to tune the occupation
fraction $f_{\rm occ}$ to bring some of the mock observations into
good agreement with observations. In this region of parameter space
one can tfind two extreme kinds of models.  One kind where the
mass interval $\Delta M\equiv \log_{10} M_{\rm max} - \log_{10}M_{\rm
  min}$ is narrow with $\Delta M<1.0$ dex, others where the mass interval is 
extended $\Delta M>1.0$dex going up to the maximum halo mass
present in the simulation at that redshift, up to $\Delta M = 2.5$ dex
in some cases.  
 
The third region in parameter space corresponds to $\log_{10}M_{\rm
  min}<10.5$. In this case only models with a very narrow mass interval of
at most $0.5$ dex ($\log_{10}M_{\rm max}<11.0$) and low
occupation fractions $f_{\rm occ}<0.3$ are allowed. 

Without any additional information our method allows us to infer that
most of the successful models are found in the second and third regions of
parameter space. This result was already expected from halo
abundance calculations shown in Figure \ref{fig:halos} and discussed
in the previous subsection. 

However, the additional information we gain with this test is the
relative abundance of models in the parameter space. Not all the
models in the second region have an equal abundance. By inspection of
Figure \ref{fig:landscape} it seems that models with $\log_{10}M_{\rm
  min}\sim 10^{11.3}$\hMsun and low occupation fraction $f_{\rm}<0.3$
are preferred.  In the next sub-section we explore in detail the
models in this region, imposing tighter constraints on the KS test
results and exploring the mocks' consistency with the angular correlation
function.  

\begin{figure}
\begin{center}
\includegraphics[width=0.95\linewidth,angle=0]{./plots/Fig4_match_P5.pdf}
%\hspace{5mm}
%\includegraphics[width=0.46\linewidth,angle=0]{./plots/Fig4_random_P5.pdf}
\end{center} 
\caption{ Number of models with at least $N_{\rm high-P}$ mock surveys
  consistent with the observed surface number density
  distribution in terms of the KS-test values $P>0.05$. Only $\sim
  100$ models have all their mocks consistent with observations. 
  \label{fig:high_success_rate}.}  
\end{figure}
 
\subsection{Models with the largest number of consistent mock surveys}

For each model ${\mathcal M}$ there are 15 different mock surveys. In the
previous section we presented the models that had at least one
mock survey with $P>0.05$. Figure \ref{fig:high_success_rate} shows the number of models
that have at least $N_{\rm high-P}$ mocks with $P>0.05$ for both the
{\texttt  match} and {\texttt random} methods.  This shows that there
are $\sim 100$ models with all the 15 mock survey realizations with
$P>0.05$.  

We discard the models that have 14 mocks or less consistent with observations.
This cut represents a reduction by a factor of $\sim 6$ with respect
to the total number of models with at least one (1) consistent mock.

Figure \ref{fig:restriction_mock} presents the locii of these models
in the parameter space $M_{\rm min}-M_{\rm max}$ and $M_{\rm
  min}-f_{\rm occ}$. With this constraints the
number of consistent models with  $10.5 < \log_{10}M_{\rm min}< 11.0$ are
greatly reduced. This corresponds to the regions in the parameter
space in Figure \ref{fig:landscape} that already had a low number of
consistent mock surveys. On the other hand, from the right panel in
Figure \ref{fig:restriction_mock} one can see that there is not a
strong reduction on the favored values for the occupation fraction
$f_{\rm occ}$. 

\begin{figure*}
\begin{center}
\includegraphics[width=0.46\linewidth,angle=0]{./plots/Fig5_match_mass_mock.pdf} 
\hspace{5mm}
\includegraphics[width=0.46\linewidth,angle=0]{./plots/Fig5_match_f_occ_mock.pdf}
\end{center}  
\caption{Favored regions in parameter space when the constraints on
  the maximal number of consistent mocks is imposed. Only the results
  for the {\texttt match} method are shown.
  \label{fig:restriction_mock}}  
\end{figure*}



\subsection{Consistency with the Angular Correlation Function}



\begin{figure*}
\begin{center}
\includegraphics[width=0.46\linewidth,angle=0]{./plots/power_law_correlation.pdf} 
\hspace{5mm}  
\includegraphics[width=0.46\linewidth,angle=0]{./plots/power_law_correlation.pdf} 
%\includegraphics[width=0.46\linewidth,angle=0]{./plots/mmin_vs_correlation.pdf}  
\end{center}
\caption{Left: Values for the free parameters ($\theta_{0}$ vs $\beta$) 
in the fitting formula (Eq. \ref{eq:fitting}) for the angular
correlation function. Blue dots correspond to simulations and the
green cross to observations by \citet{Hayashino2004} and
\citet{Ouchi2010}. The error bars in the   theoretical data correspond
to the standard deviation from the different mocks surveys. 
%Right:
%Same results in the $\theta_{0}$-$M_{\rm min}$ plane, with the
%observational constraint on $\theta_{0}$ epresented by the red
%rectangle. Blue  squares represent models with $\log_{10} M_{\rm
%  max}<12.0$, green circles correspond to models with $\log_{10}
%M_{\rm max}>12.0$. 
\label{fig:correlation_parameters}}
\end{figure*} 


\begin{figure*}
\begin{center}
\includegraphics[width=0.46\linewidth,angle=0]{./plots/Fig6_mass.pdf}
\hspace{5mm}
\includegraphics[width=0.46\linewidth,angle=0]{./plots/Fig6_f_occ.pdf}
\end{center}
\caption{Planes $M_{\rm min}$ vs $M_{\rm max}$ (left), $M_{\rm min}$
  vs $f_{\rm occ}$ (rigth). The squares represent the models when both
  th constraints on the maximal number of  consistent mocks and the
  angular correlation function are included. 
  \label{fig:restriction_mock_and_f_occ_corr}} 
\end{figure*} 


Figure \ref{fig:correlation_parameters} shows the main results in a
$\theta_{0}-\beta$  plane where the average and standard deviation
over the mocks is shown in comparison with the result derived from
observations.  The error bars in this Figure represents the standard
deviation of the ACF over all the sub-fields in the 15 mock
observations. These error bars are larger than the statistical
uncertainty from the fitting procedure on a single field. 

In the left panel Figure \ref{figure:correlation_parameters} we see
how the observational ACF measured by \cite{Hayashino2004} (green dot)
is successful in reducing the total number of possible models. Only
those with angular-correlation length within
$15^{\prime\prime}<\theta_{0}<23^{\prime\prime}$ are considered to
reproduce observations.     {\bf Cual es el resultsdo de Hayashino
  para $\theta_0$?}

In Figure \ref{fig:restriction_mock_and_f_occ_corr} we present the preferred
models in the planes $M_{\rm min}-M_{\rm  max}$ and $M_{\rm min}-f_{\rm occ}$
for the {\tt match} method. To make this selection we consider a that
a model is consistent with observations if there is a $1-\sigma$
overlap between both the correlation length $\theta_0$ and the power
$\beta$ in the correlation function. Comparing this Figure with the
results shown in Figure \ref{fig:restriction_mock} we see that the
models that were removed were those with highest minimal masses
$M_{\rm min}>10^{11.2}$, for all the values of $M_{\rm max}$, which
had preferentially high occupation fractions of $f_{\rm occ}> 0.5$. 



%In order to better illustrate the results from this test we
%show in the right panel of Figure \ref{figure:correlation_parameters} 
%the plane $\theta_{0}$-$M_{\rm min}$ for all models divided in two
%disjoint sets. The first are the models $\log_{10} M_{\rm
%    max}<12$ (blue dots), the second corresponds to $\log_{10} M_{\rm
%    max}>12$ (green dots). The colored rectangle includes
%the parameter region which is consistent with the observational
%constraint in $\theta_{0}$.  With this restriction we find that most

%models with $\log_{10} M_{\rm max}>12$ and  $\log_{10} M_{\rm min}>11.s$ 
%can be safely ruled out.  





\section{Discussion}

Matching the galaxy surface number density statistics to
our mock  data sets the median
mass of all successful models in the range $(10^{10}-10^{12})$
\hMsun. Including more strict criteria on the number of mock surveys
that must be consistent with those observations and additional
information from the angular correlation function greatly reduces the
number of possible models. We end up with $\sim 50$ models out of the
initial 9000 possible combination of parameters.  

We find that to make a physical interpretation of these models it is
useful to do it in terms of the size of halo mass range. We now use the
variable $\Delta M=\log_{10}M_{\rm max} - \log_{10}M_{\rm  min}$, together
with the occupation fraction $f_{\rm occ}$ to help us build a classification
of all the successful models into three families: 


\begin{itemize}
\item[(1)] Low $f_{\rm occ}\leq 0.3$ and low $\Delta M\leq 1.0$
  dex: 24 models.
\item[(2)] Low $f_{\rm occ}\leq 0.3$ and high $\Delta M > 1.0$dex: 11
  models
\item[(3)] High $f_{\rm occ}> 0.3$ and low $\Delta M\leq 1.0$: 14 models
\end{itemize}

There is a clear majority of models with a narrow $\Delta M\leq
1$, compared to the $2.5$dex of halo mass available for occupation at
that redshift. Such models imply that there is a cut at lower and higher halo
masses that render inefficient the presence and/or detection of LAEs.

At the low mass end, such cut can be readily interpreted in terms of the
minimal halo star formation rate needed to produce the necessary \ly 
luminosity to be above a given detection threshold.  However, under
the reasonable assumption of star formation rate increasing with halo
mass, the cut at higher halo masses requires a different
explanation. There are two different physical scenarios to explain
this. 

One possible interpretation can be made in terms of a decreasing escape
fraction of \ly radiation in massive systems. There are detailed models for
radiative transfer that support the idea that massive galaxies with
higher metallicities have larger dust contents and less concentrated
ISM than lower mass systems which, due to the resonant nature of the
\ly line that increase the probability of absortion  of\ly photons, produce
high absorption of \ly photons but not of continuum or
other n-resonant lines \citep{Laursen2009,ForeroRomero2011}.  

A second scenario is that larger systems have more extended gaseous
envelopes which, due to resonance effects of the \ly line, leading to
a low surface brightness and a broader line, making these systems less
observable in narrow band filter surveys \citep{Zheng2010}. 

The preference for narrow $\Delta M$ ranges to hosting LAEs together
with the minor presence of reasonable models with $M_{\rm max}>
10^{12}$\hMsun \  shows that our models support theoretical and observational
insights where the most massive systems are not bright \ly sources
\citep{ForeroRomero2012}, \citep{Shapley2003}

\subsection{Comparison against results from blind surveys}

We also find that $70\%$ of the best models are found in families (1)
and (2), with a low occupation fraction $f_{\rm occ}\leq 0.3$. This
preference goes in the same direction as the observational constraint
on the escape fraction $f_{\rm esc}\sim 0.1-0.2$ derived at $z=2.2$ by \cite{Hayes2010}
and recent theoretical study of observational data in a wide redshift
range $0<z<6$  \citep{Dijkstra2013}.  

The observational estimation by \cite{Hayes2010} was based on blind
surveys of the H$\alpha$ and Lyman $\alpha$ line. Using corrections by
extinction to obtain an estimate for the intrinsic H$\alpha$
luminosity, and using values for the theoretical expectation of the
ratio Lyman$\alpha$/H$\alpha$ they derive an bulk escape fraction for
the Lyman$\alpha$ radiation of $f_{\rm esc}=(5.3\pm 3.8)\%$ or $f_{\rm
  esc}=(10.7\pm 2.8)\%$ if a different dust correction is used. 

They also showed that that the luminosity function for LAEs at $z=2.2$ is
consistent with the escape fraction being constant for every galaxy
regardless of its luminosity. From this results they derive that
almost $90\%$ of the star forming galaxies emit insufficient
Lyman $\alpha$ to be detected, effectively setting the occupation
fraction to be $f_{\rm occ}=0.10$.  

\cite{Dijkstra2013} used a similar principle to derive their results. They
compared observationally derived star formation functions to LAE
luminosity functions. At $z\sim 3.0$ they derive an effective escape
fraction of $f_{\rm esc}=(17\pm 5)\%$ could be interpreted as an
occupation fraction $f_{\rm occ}\sim 0.2$.  We consider a success of
our method the fact that we find that most of the consistent models
show a low occupation fraction.    



\subsection{Comparison to other clustering estimates}

Observational based on the ACF inferred from photometric measurements
in the Extended Chandra Deep Field South have shown that the median
dark matter masses of h los hosting LAEs is $\log_{10}M_{\rm
  med}=10.9^{+0.5}_{-0.9}$\Msun, with a corresponding occupation
fraction of $1-10\%$  \citep{Gawiser07}.  Our results are in a general
good agreement with those estimates for the host mass. This is not
completely unexpected given that we have also required consistency
with ACF measurements.   

The novelty in our approach is that we have a detailed estimate for 
host halo mass range together with the escape fraction. This allows us
to demonstrate that the halo mass range could be very narrow $\Delta M <
0.2$dex, something that cannot be inferred from ACF analysis
alone. 

We also find interesting that our ACF analysis is also not enough to rule
out models with a high occupation fraction $f_{\rm occ}>0.3$, which
represent almost one fourth of our best models, coinciding with a wide
range in halo masses $\Delta M>1.0$ dex. These models can only be
considered unfavorable based on the density and abundance observations
as we have described in the previous subsection.


\cite{Ouchi2010} presents an analysis of LAE observations in the
redshift interval $3.1<z<7.0$. At $z=3.1$ They get to an average
mass for the host dark matter halos of $M_{h}=2.9^{+24.0}_{-2.9}\times
10^{10}$ \hMsun with a corresponding duty cycle of $0.008\pm
0.03$. This broadly matches the expectations from the first family of
models, also summarized in Table \ref{table:firstfamily}. In
particular it seems to favor only the very low occupations fractions.  




\subsection{In the context of abundance matching models}


Considering the additional evidence for a low escape fraction and 
assuming a direct relation between $f_{esc}$ and $f_{occ}$ we can
say that the preferred models are in families (1) and (2) with a
clear majority composed by those with narrow mass range.

Abundance matching methods throughout cosmic time from redshifts
$0<z<8$ \cite{Behroozi2013a,Behroozi2013b} work based on observational
results for Lyman Break Galaxies. In those studies the minimum halo
mass considered to be relevant in their analysis is
$10^{11.4}$\hMsun. For that mass, \cite{Behroozi2013a} report a
stellar mass mass around $(1.0\pm0.3)\times 10^{9.0}$
\hMsun, while their star formation rate is in the range $0.6\pm 0.2$
Msun yr$^{-1}$, which nevertheless are close to the lower bound of
values inferred for LAEs at high redshift
\citep{Gawiser2007,Nilsson2009,Pentericci2009}.  





In our results, all the preferred models have a halo mass range lower
than the minimum of $M_{\rm min}<10^{11.4}$\hMsun considerend in
abundance matching at $z=3$. This mass scale is also strictly superior to
the great majority of  $M_{\rm max}$ values allowed in the models with
low escape fraction.   This suggests that a detailed and careful study
of the spectral and photometric properties of LAEs coupled to the kind
of analysis performed in this paper can be a guide in the study of the
properties of low mass dark matter halos at $z=3.1$, extending the
capabilities of abundance matching methods.


\subsection{On the reproducibility of our results}

... All the software to produce the results in this paper is publicly
available. 

... The raw catalogs can be obtained from the MultiDark database but
can also be obtained in the repository of this paper on github.

\section{Conclusions}
\label{sec:conclusions}

In this \documentname we look for constraints on the preferred mass
and occupation fraction of dark matter halos hosting Lyman Alpha Emitters at
redshift $z=3.1$ in a $\Lambda$CDM cosmology. We perform this study
paying attention to the impact of cosmic variance in these results. We
build a large number of mock catalogs matching observational
geometries. The mocks are constructed from a N-body simulation
following a simple recipe to assign a single LAE to each halo. Only
a fraction $f_{\rm occ}$ of halos with a mass range  $M_{\rm
  min}<M_{\rm h}<M_{\rm   max}$ can host a LAE. We proceed with a
thourough exploration of the space of free parameters $M_{\rm min},
M_{\rm max}$ and $f_{\rm occ}$ to find mocks that are consistent with
two observational constraints: the surface number density and the
angular correlation function. Out of the initial $9000$ initial
combinations of parameters in the model we find $49$ arrangements
consistent with observations.


The most relevant result from our study is the following. The 
successful parameter arrangements present a wide range of possibilities for
DM hosting LAE hosts at $z=3.1$. This means that the available LAEs' spatial
information is not sufficient to put a tight constraint on the mass
and occupation fraction of DM halos hosting LAEs. In particular it is
not possible to put a strong constraint on the occupation
fraction. Such a wide range in solutions is facilitated by the large
dispersion in the statistics derived from the mocks. All the halo mass
ranges and occupation fractions deduced in previous analysis in
previous analysis \citep[i.e.][]{Gawiser2007} are a subset of the
models we find in this \documentname.  
 

Nevertheless, the sucessful models can be split into three families
depending the mass range $\Delta M=\log_{10}M_{\rm max} -
\log_{10}M_{\rm min}$ and the occupation fraction $f_{\rm occ}$. The
first family is narrow both in $\Delta M$ and $f_{\rm occ}$, a second
family is only narrow in $\Delta M$ and the third is narrow in $f_{\rm
  occ}$ and broad in $\Delta M$. 

 
A second results from our investigation is the existence
of a large number of models with a very narrow mass range $\Delta M<
0.1$ dex that are consistent in every aspect with the spatial
distribution of observed LAEs. In order to have these characteristics
it is required that halos around the mass $M_{\rm   max}$ for each
model become inefficient in hosting detectable LAEs. Two possible
ways to achieve this is having LAEs with a decreasing \ly escape
fraction with increasing mass or increasing screening effects by
neutral Hydrogen around the most massive systems
\citep{Laursen2009,ForeroRomero2011}.   

A third aspect that stands out is the existence of a dominant family
of models with narrow $\Delta M<0.1$, as described before, and also a
low occupation fraction $f_{\rm occ}<0.3$. This gives support to the
common wisdom stating that only a small fraction of star forming
galaxies can be detected as strong LAEs. 

We foresee that the new observations with new instruments (such as MUSE,
Hyper SuprimeCam and HETDEX) covering larger fields and a wider range
of luminosities will be key in imposing tighter constraints on the
properties of dark matter halos hosting LAEs.  


\section*{Acknowledgments} 
J.E.F-R thanks the hospitality of Changbom Park and the Korea
Institute for Advanced Study where the first full draft of this paper
was completed. The authors also thank Peter Laursen, Paulina Lira and
Alvaro Orsi for helpful comments on the physical interpretation and
the presentation of our results. 



\begin{table}
\begin{center}
\begin{tabular}{ccc}\hline\hline
$\log_{10}M_{\rm min}$ & $\log_{10}M_{\rm max}$ & $f_{\rm occ}$\\\hline
10.1& 10.2& 0.1\\
10.3& 10.5& 0.1\\
10.4& 10.7& 0.1\\
10.5& 10.6& 0.3\\
10.5& 10.9& 0.1\\
10.5& 11.0& 0.1\\
10.6& 10.8& 0.2\\
10.6& 11.2& 0.1\\
10.6& 11.3& 0.1\\
10.6& 11.4& 0.1\\
10.6& 11.5& 0.1\\
10.6& 11.6& 0.1\\
10.7& 11.0& 0.2\\
10.8& 11.2& 0.2\\
10.8& 11.3& 0.2\\
10.8& 11.4& 0.2\\
10.9& 11.3& 0.3\\
10.9& 11.7& 0.2\\
10.9& 11.8& 0.2\\
10.9& 11.9& 0.2\\
11.0& 11.6& 0.3\\
11.0& 11.7& 0.3\\
11.0& 11.8& 0.3\\
11.0& 12.0& 0.3\\\hline\hline
\end{tabular}
\end{center}
\caption{\label{table:firstfamily}List of parameters for the first
  family of models. Narrow   mass range $\Delta M\leq 1.0$dex and low
  occupation fraction $f_{\rm occ}\leq 0.3$.} 
\end{table}



\begin{table}
\begin{center}
\begin{tabular}{ccc}\hline\hline
$\log_{10}M_{\rm min}$ & $\log_{10}M_{\rm max}$ & $f_{\rm occ}$\\\hline
10.6& 10.7& 0.4\\
10.7& 10.8& 0.5\\
10.8& 10.9& 0.6\\
10.8& 11.0& 0.4\\
10.9& 11.0& 0.8\\
10.9& 11.0& 0.9\\
10.9& 11.1& 0.5\\
11.0& 11.1& 1.0\\
11.0& 11.4& 0.4\\
11.1& 11.3& 0.7\\
11.1& 11.7& 0.4\\
11.2& 11.5& 0.7\\
11.2& 11.6& 0.6\\
11.2& 11.9& 0.5\\\hline\hline
\end{tabular}
\end{center}
\caption{\label{table:secondfamily}List of parameters for the second family of models. Narrow
  mass range $\Delta M\leq 1.0$dex and high occupation fraction $f_{\rm occ}>0.3$.}
\end{table}






\begin{table}
\begin{center}
\begin{tabular}{ccc}\hline\hline
$\log_{10}M_{\rm min}$ & $\log_{10}M_{\rm max}$ & $f_{\rm occ}$\\\hline
10.6& 11.7& 0.1\\
10.9& 12.1& 0.2\\
10.9& 12.2& 0.2\\
10.9& 12.3& 0.2\\
10.9& 12.4& 0.2\\
10.9& 12.5& 0.2\\
10.9& 12.6& 0.2\\
10.9& 12.7& 0.2\\
10.9& 12.8& 0.2\\
10.9& 12.9& 0.2\\
10.9& 13.&0 0.2\\\hline\hline
\end{tabular}
\end{center}
\caption{\label{table:thirdfamily}List of parameters for the third family of models. Broad
  mass range $\Delta M >1.0$dex and low occupation fraction $f_{\rm occ}\leq0.3$.}
\end{table}


\bibliographystyle{mn2e}
\bibliography{references} 

\end{document}

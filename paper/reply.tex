\documentclass[12pts]{article}
\begin{document}
Dear Editor, \\

We wish to thank the referee for a detailed report on our paper. \\

The referee expressed 5 major points of concern. Some of them were due
to unclearly written sections, other made reference to issues with the
observational data and choices to perform the comparison against the
simulated data.\\ 

In this new version we respond to all these concerns. We have
clarified the text where it was needed and performed new tests on the
data. We have also corrected an error in the computation of the
angular correlation function, strenghtening our conclusions. Therefore
we resubmit this draft to be considered for publication in the MNRAS.\\ 

We wish to emphasize that our paper uses a new method in the context
of the study of LAE galaxies. Correspondingly, the results are novel
with respect to the already published literature.\\

In what follows we reply to each one of the major points raised by the
referee. The comments by the referee are in boldface. Our reply
follows to each of these comments.\\


Best regards,\\

Jaime E. Forero-Romero and Julian Mejia-Restrepo\\


{\large \bf Major points}\\

\begin{enumerate}
\item {\bf As the authors are aware, each observational survey for LAEs has
  different selection effects, and some use very different thresholds
  in equivalent width.  Hence you need to use the number density and
  autocorrelation function (ACF) of LAEs from the *same* survey for an
  investigation like this.  You could try to get Yamada's LAE
  positions to calculate the ACF, since you appear to prefer that
  survey for number density estimation.  Or you could pursue joint
  analysis of number density and ACF for smaller surveys e.g.,
  Ouchi's. }\\


  We agree with the referee that the two statistical tests should be
  performed on the same data set. Unfortunately neither the galaxy catalog
  nor the ACF from Yamada et al. 2012 are available. We contacted
  Dr. Yamada asking for information about the ACF, but we didn't get
  any reply.  

  The currrent version of the draft features a comparison with the ACF
  from a survey with an average number density from Hayashino et al
  2004. 

  On the other hand, Using a small field as Ouchi's for our analysis
  is not feasible, given that our method is based on the analysis of
  the surface density \emph{distribution among different}
  fields. Having the distribution data is central to discard a large
  fraction of models (Section 3.3).


\item {\bf 
  The major result claimed that all models have a narrow mass range
  of halos hosting LAEs is guaranteed by the choice of a minimum
  value of $f_{\rm occ}$ of 0.1.  This should be obvious to the authors, as
  there is a degeneracy between $f_{\rm occ}$ and the mass range of halos
  that you have hidden by forcing $f_{\rm occ}$ to values larger than those
  previously claimed in the literature. }


  As the referee highlights, we have found a new set of acceptable
  parameters for the escape fraction than those that have been
  previously claimed by the literature.

  However, we do not find a clear evidence for a degeneracy between
  the occupation fraction $f_{\rm occ}$ and the mass range $\Delta
  M$. There is, however, a clear degeneracy between the minium halo
  mass $M_{\rm min}$ and the occupation fraction. 

  The reason for that trend is easy to understand. In LCDM the halo
  number density $n$ is verysensitive to the mass. The dependency is
  roughly $n\propto  M_{h}^{-1}$. Therefore, at a fixed minimum halo
  mass, increasing the mass range by 1.0 dex barely raises the number
  density by  
  $0.10$. Increasing the mass range by 2.0 dex  raises the number
  density by $0.11$. This explains that we cannot find any clear
  dependence between $\Delta M$ and the $f_{\rm occ}$. On the other
  hand, decreasing the minium mass by 1.0 dex would increas the number
  density by a factor of {\bf $10$} (!). Requiring that models with
  small values for the minium mass to have a small occupation
  fraction.  
  
  This trend is clearly seen on the right panel of Figure 3,4 and
  6. This implies that exploring very low escape fractions values of
  $0.01$ would require the exploration of halo masses aroud the range
  of $10^{9}$ Msun or   below, which is lower than the minimum halo
  masss of $10^{9.5}$ Msun   that is confidently resolved in the
  simulation.  

  \item {\bf 
The authors cite similar work by Walker Soler et al. 2012 but fail
   to mention in the introduction how the current paper has the same
   goals as that earlier work.  If applied more carefully, their
   approach should be able to improve on the earlier work, but it is
   important to be clear about how similar the papers are and how the
   current one hopes to improve on the previous one. 
}

The work of Walker-Soler 2012 actually does not pursue a similar goal
as our paper. They take a fixed set of results of a semi-analytical
model to infer the mass of the \emph{descendants} from LAEs at
$z=3.1$. We have removed the reference to that work that might show
that we are improving work they have been done, because that's not the case.

\item {\bf 
Comparing the most heavily clustered of 15 mocks against SSA22 is
  unfair. SSA22 is a ~5 sigma overdensity, which is much rarer than
  1/15. 
  }
  
We agree with the referee that a comparison of a $\sim 5 $ overdensity
with an average field would not be fair. Therefore we have removed
that constraint from the results in the new version of the paper. Now
we only include the ACF comparison against the results of an average
density field from Ouchi 2008. 

Related to this point we have added the left panel in Figure 1 and
subsection 2.2 to discuss the adequacy of including SSA22 as an
observational benchmark in the number density tests.


\item {\bf 
  There seems to be significant confusion between halo occupation
   fraction and Lyman alpha escape fraction.  Only in an incredibly
   simplified model where 100\% of Ly alpha photons escape from LAEs
   and 0\% escape from non-LAEs would the comparison in section 4.1
   make sense.  In a more realistic model, it is not clear how this
   investigation could hope to comment on the escape fraction of Ly
   alpha photons. 
}\\

We agree with the referee in this point.

We have corrected all the mistakes where we meant occupation
fraction but wrote escape fraction. We have alsoremoved the discussion
comparing our results to escape fraction estimates, in spite of the
effort of other authors to link these two different quantities.  

\end{enumerate}

Additional suggestions:
\begin{itemize}
\item{\bf Only allowing one LAE per dark matter halo is a simplistic model.  You
could try more complicated HOD models where $f_occ(M)$ is not constant
to see if they make a significant difference in the results.  But
arguing against this by citing the Jose et al. 2013 paper seems
reasonable.  You could also note the lack of close-proximity LAE pairs
noted in papers led by Nick Bond as further evidence against multiple
LAEs occurring in the same halo. }

We would like to add a reference to the work of Nick Bond, but failed
to fine what exact paper makes reference to the lack of
close-proximity LAE pairs.

\item{\bf You could explore a wide range of $f_occ$ down to properly tiny values
such as 0.001 without increasing the run time by indexing it
logarithmically.}

As we mentioned before in the Reply to Major Point 2, tiny values of
0.001 can only produce reasonable models for very low mass halos. This
is due to the contraint on the number density. An occupation fraction
of 0.001 requires models with a minimum mass on the order of $M_{\rm
  min}=10^{9}$ solar masses, which is well below the mass resolution
of the simulation we use. This comment has been added to the paper. 


\item {\bf Be careful when turning filters into cosmological volumes - you should
not simply use the FWHM.  See Gronwall et al. 2007 for a detailed
discussion of this.}

We have included a paragraph about this point in Section 4.4.

\item {\bf Be careful when comparing results for halo occupation fraction for
LAEs from a survey with a 190A observed EW threshold to the more
typical 20 A EW rest-frame, which equals 80A observed at z=3. }

We have included a paragraph about this point in Section 4.4

\item {\bf SSA22 is known to be an extraordinary overdensity, as you note.  The
overdense region, and possibly the entire field, will bias the
measured number density of z=3.1 LAEs upwards.  So one should be very
cautious including this field in a study like this one. }

We have included subsection 2.2 to comment on this point.

\item {\bf When using a constant value for $f_{\rm occ}$, you could use just
  $M_{\rm min}$, $M_{\rm max}$ as parameters, choose the value of
  $f_{\rm occ}$ that fits the observed number density (propagating
  uncertainties) and then test the resulting ACF versus data.  This
  would be accurate because a random sub-sample of halos has the same
  clustering as the full sample. } 

  We agree that this is a feasible approach. The emphasis we meant in
  our approach was the effect of cosmic variance on the size of
  observed fields. This is why we include the tests on all sub-samples
  and not just one.

\end{itemize}

\end{document}
